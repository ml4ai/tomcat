

\chapter{Question-Asking and Goal Inference}
\label{ch:question_plan}
\textbf{Salena Ashton, Stephen Kim, Loren Champlin, Liang Zhang, Clayton Morrison}

\section{Introduction}

As people interact with each other, they display observable behaviors (speech, body language, expressed emotion, or interaction with the environment) that may suggest unobservable behaviors (thoughts, feelings, plans, or beliefs). The process of inferring these unobservable behaviors is called Theory of Mind (ToM) and is often studied through the analysis of body language, spoken natural language, human development, social cultural differences or non-neurotypical human cognition. When humans ask questions to obtain information, they most likely do so with an intent, a plan, or desire.

The term planning in Artificial Intelligence research does not necessarily have the same meaning as it would with human ToM. Within AI, a classical plan, also called flat, is a sequence of states and actions that an agent can execute to reach a goal. These plans, composed of simple actions, can be represented with a flat data structure such as a list. As plans become more complex or abstract, require constraints, or have multiple levels of abstraction within the same plan, they are best represented by a hierarchical task network (HTN), which is a tree of possible plans.

Research Question 1: How do spoken questions reveal another person’s plan or intent?
We will investigate whether verbalized question-asking can infer a human’s ToM when uttered in a simulated search-and-rescue (SAR) scenario within the Minecraft environment, designed by ASIST.

Research Question 2: Can those who listen to these questions accurately decide if the plan is simple, sequential, or hierarchical?

The results of this investigation will guide further research into the broader research question: do questions have hidden goals that infer plans that are flat or hierarchical? When are uttered questions best represented with classical planning or HTN data structures?  Can the distinction of such structures improve predictive performance for Automated Social Intelligence (ASI) agent intervention.


Existing literature about HTNs, ToM, psycholinguistics and question-answering
are abundant, as are papers about question generation for information
retrieval. Fewer papers exist about question-asking and even fewer papers focus on the intersection of ToM for human beings and the HTN mapping of dialog acts. Hawkins and Goodman \citet{hawkins2017you} approach question-asking as evidence of the questioner’s hidden goals. They categorize questions into binary question-answer pairs, pragmatic question-answer pairs where the most socially-acceptable response does not strictly answer the literal question asked, and probabilistic question-answer pairs, where the most probable answer is the expected value of a set of likely answers.

Within the SAR scenario of ASIST’s Minecraft environment, human players will
verbalize plans, suggestions, or ideas to each other. Sometimes these plans or
requests are fulfilled and other times they are violated, forgotten or
disregarded. Baldoni et al \citet{baldoni_2018} represent social interactions of AI agents with first-order logic; this representation includes question-asking, termed commitment initiation. This representation can also express action sequence and constraints of actions and plans.



\section{Approach}

We assume that questions have hidden goals that infer plans. As teams ask more questions of each other, human team ToM converges toward cooperative behavior.
We will investigate whether question-asking is associated with team planning, defined as a set of goals, strategies, or tasks that are executed. We define coordination as behaviors and utterances to create a common plan or strategy. We further define cooperation as team behaviors that implement an already-agreed up on plan.

Two human annotators will code all uttered questions between teammates who are
tasked with the mission search for, locate, and diagnose victims in a Minecraft
SAR scenario. We use the qualitative coding procedure known as Grounded Theory,
as defined by Corbin and Strauss, \citet{corbin_strass_2015}. More
specifically, and as defined by Saldana, \citet{saldana_2015}, we will use a Grounded Theory hybrid of In-vivo Coding for action behaviors, Process Coding for a state or action across some interval of time, and Causal Coding theory for categorical coding procedures.

In cases where questions have co-reference or other contextual dependencies, only that direct dependency will be coded for local semantic meaning.

\subsection{Data}

Label development for this pre-registration investigation will come from the annotations of ASIST Study 3 Spiral 2 pilot video observations: 419, 420, 421, 422, 423 and 424. Due to the expensive costs of human annotation and strict adherence to grounded theory methodology, this experiment is limited to six videos.  Future testing of these labels will be on the real data, released after 29 March 2022. These six pilot videos are of three distinct teams that play two missions each.  

Two human annotators will generate concept-level labels, which are defined as the smallest pieces of data that can be used to explain an action, interaction, condition, context, or consequence of having uttered a question. Data analyses will not be performed over the raw data from the videos; they will be performed over the concept-level labels. These labels are not pre-determined, mapped to any type of human plan, thus giving a more solid, grounded analytical meaning to any emergent categories. 


After the generation of these concept labels, we will investigate the connectivity and causality of each label to discover possible relationships and similarities. Densely-connected concept labels will suggest themselves as subcategories and categories. Sparsely-connected labels will not be discarded; they will be used to evaluate variability within patterns and categories that emerge. As subcategories and patterns emerge, they will describe the presence or absence of team actions, interactions, conditions, and consequences of question-asking. 

We make the following considerations when creating codes: 
\begin{itemize}
    \item Frequency will not dictate importance, causality, or connectivity of a concept
    \item Labels will be stemmed and minimally normalized 
    \item How do the labels explain the rate of question-asking across time, between any subset of a team, between the same team across the two different missions, and across all six videos? Does this explanation infer predictibility power?
\end{itemize}

When all concepts, subcategories, categories can reasonably explain the phenomena of the video observations, one or two super-categories or theories of team plan will emerge. Though we have a priori assumptions as to what these team plans could be, we do not presume them. Categories that emerge from denser concept connections are currently assumed to have greater predictive power and ToM inference potential.


\section{Evaluation}

Because of the small sample size of this investigation, we do not evaluate at this time. Instead, we investigate correlation of concept-level labels through data visualization. Below is a list of possible visualizations for the investigation of a specific observation:
\begin{itemize}
    \item Connectivity of concept-level labels: radial diagrams, arc diagrams, matrix diagrams or graph networks
    \item Frequency patterns of words or concept labels (normalized, word count / total number of words in that question): scatterplots or histograms
    \item Correlation of words and labels with time: time series, scatterplots. 
    \item Concept-level subcategorization(s): clustering, PCA (concept labels possibly projected onto subcategorical spaces), or hierarchical visualizations.
\end{itemize}

Future research beyond this proposal will address the TA3’s Process Measure ASI-M5: Coordinative Communications to measure teamwork, which will include additional video observations for real data in Study 3, and continued investigation of whether human plans and ToM are best represented by classical planning or HTN planning. 

Future measures of evaluation may include:
\begin{itemize}
    \item Reproducibility: Are the labels and categories explainable and verifiable? 
    \item Do the sets of concept-level labels have at least a Kappa Cohen score of 0.8 between the two annotators?
    \item Precision and recall metrics of the concept-level and category patterns:
    \begin{itemize}
        \item To describe human ToM and plans
        \item Generalizability for real data with no ASI interventions
        \item Generalizability for real data with higher ASI interventions.
    \end{itemize}
    \item Variation of Pattern Development: Quantification of pattern variance 
\end{itemize}


Given the categorical nature of label development, usefulness and evaluations of these visualizations will serve as a preliminary indicator of whether questions are best represented by flat or HTN structures. Results from these visualizations will guide continued research into human ToM and human plan representation.

