\chapter{Closed-loop communication detection}
\label{ch:clc}
\textbf{Yuwei Wang}

\section{Introduction}

\section{Approach}

We are going to build a rule-based dialogue agent to detect Closed-loop communication events and test the agent using the TA3 dataset.
The existing ToMCAT dialogue system is currently able to analyze spoken
conversations in real-time, to extract entities and events of interest with a
powerful rule-based framework \citep{valenzuela-escarcega-etal-2016-odins}, classify
dialogue acts, and detect sentiment. To extend this system and detect
closed-loop communication, we are going to create a separate worker and detect
the relevant labels based on the three phases of closed-loop communication
within several speech turns, namely, the call-out, check-back, and
closing-the-loop phases \citep{Hargestam.ea:2013}. For example, we can
navigate the call-out phase with Odin labels like ``HelpRequest", and then look
for the ``Move" label in the next 5 speech turns, if we can detect this
check-back phase, keep on looking for additional Closed-loop phase with an
``Acknowledge" label, and return a ``CallOut-CheckBack-Close" extraction. If the
Closed-loop phase is not detected at all, a “CallOut-CheckBack” extraction can
be returned. If both phase 2 and phase 3 are not detected, exit the detecting
loop after 5 steps from the call-out phase and return ``OpenLoop" extraction.

We will firstly implement this agent with a small set of test data, extracted
from the TA3 dataset, and then evaluate the performance of this agent for its
precision, recall, and F1 score, and then explore whether data-driven methods
could improve the score of its performance.

\section{Evaluation}

The agent of the Closed-loop communication detector will be evaluated with a
subset from the TA3 transcripts. Human annotators will be trained to detect the
3 phases of closed-loop communication, and whether the loop is closed or not.
The interrater reliability of the annotators will be measured using Cohen’s
kappa. When the percentage of agreement between annotators reaches 80\%, and K
> .70, annotators could start work on the formal annotation of the data. The
precision, recall, and F1 score will be used to evaluate the performance of our
primary closed-loop communication agent. When improvement on the agent applied,
we can further compare the performance of our primary rule-based model with the
improved model of rule-based and data-driven combined.
