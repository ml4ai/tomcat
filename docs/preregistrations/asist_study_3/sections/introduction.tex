\chapter{Introduction}
\textbf{Adarsh Pyarelal}

\section{Purpose and structure}

This document serves the following purposes.

\begin{itemize}
    \item Declare the capabilities we aim to demonstrate for our online agent
        components in ASIST Study 3.
    \item Declare the offline analyses we plan to perform on ASIST study 3 data
        in advance, with as much specificity as we can muster.
\end{itemize}


The goal of the preregistration process as originally devised
\citep{Nosek.ea:2018} is to separate hypothesis-generating (exploratory)
research from hypothesis-testing (confirmatory) research. A large portion of
the activities engaged in by TA1 performers in ASIST - e.g., development of
models, algorithms, and systems - does not fit neatly into the paradigm of
hypothesis generation and testing.  For this reason, while we endeavor to
specify our analyses and capabilities in as much detail as we can, we do not
specify specific social science-style hypotheses to be tested\footnote{Whether
TA1 performers \emph{should} be specifying formal, quantitative hypotheses is
another discussion.}.

One of the primary motivations for this preregistration document from the
perspective of a TA1 performer team is to accelerate the writing up of
manuscripts for publication by using the structure provided by the
preregistration process to plan ahead for publications.

Each of the sections in this document is a `component preregistation' that
corresponds to a publication `seedling', with the content optimized for what we
call `copy-pasteability', that is, the ability to be copied verbatim into a
manuscript for publication with minimal changes. This is intended to reduce
duplicate effort between writing up preregistrations and publications.

Since our team is relatively large, there is little overlap in the author lists
for the individual component preregistrations. Thus, each component
preregistration has the names of the primary authors responsible for its
content section displayed under its title heading in addition to the table at
the beginning of this document. Authorship is ascribed to those who have
contributed substantially to the ideation or writing of the content.

\section{Overview}

Broadly speaking, we are aiming to develop a suite of open-source technologies
for artificial social intelligence, with a focus on computational understanding
of spoken team dialogue. Each of the component preregistrations demonstrates a
capability that we believe is important for artificial social intelligence, and
is currently integrated or will be integrated in the near future into our ASI
agent that will be evaluated in ASIST study 3 or future ASIST experiments.

For ASIST Study 3, we will deploy three Dockerized components. Two of them can
be classified as `analytical components' in the parlance of the ASIST program,
and one as an `ASI', i.e. an AI agent imbued with artificial social
intelligence. The current delineation between ACs and ASIs has been made based
on container boundaries. In principle, the two ACs devoted to speech and
natural language processing (see \autoref{fig:nlp-architecture}) could be
bundled and deployed as part of our ASI, but we opt to keep them separate as we
expect that the resulting modularity will be beneficial for future
applications.

\begin{figure}
    \centering
    \includegraphics[width=6.5in]{images/nlp_architecture}
    \caption{Architecture of our multi-participant dialogue analysis system.}
    \label{fig:nlp-architecture}
\end{figure}

%Give a brief summary of the preregistrations and how they contribute to the overall architecture.

The component preregistrations in this document span a broad spectrum of
research topics, but work together to form a cohesive suite of ASI
capabilities. 

% TODO complete this paragraph
