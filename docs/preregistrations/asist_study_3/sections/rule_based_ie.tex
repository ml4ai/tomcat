\chapter{Rule-based entity and event extraction}
\label{ch:rule_based_ie}
\textbf{Remo Nitschke, Adarsh Pyarelal}

\section{Introduction}

For humans, verbal and non-verbal communication are indicators that provide
significant insight into social processes. Within the scope of the ASIST
program, team-building and communication within teams is of heightened
interest. We assert that understanding the semantics of verbal communication is
essential to develop a model of human individuals and teams. Our `DialogAgent'
AC is designed with the goal of extracting information from player dialog that
can help build and update a model of players' mental states. The outputs of the
DialogAgent AC are consumed by the ToMCAT ASI (\autoref{ch:pgm}).

Raw natural language data is messy by nature - in order for a machine to reason
about its semantics, it must be processed to extract structured information.
In order to do this, the DialogAgent leverages Odin
\cite{valenzuela-escarcega-etal-2016-odins}, a powerful rule-based information
extraction system developed at UArizona.  While much of modern information
extraction and event extraction is done using neural networks
\cite{Ahmad2021GATEGA, Du2020EventEB}, we opt for a rule-based system for the
following reasons:

\begin{enumerate}

 \item It allows us to be more flexible with our extraction labels. We can
     quickly add or remove labels if we see the need to do so.

 \item It allows for high precision for the labels we are interested in. Rules
     can be crafted to be precise (albeit at the cost of recall).

 \item Rules do not require us to create, maintain, and annotate extensive
     datasets. This is especially pertinent to the ASIST program, with the
     relatively fast cadence of changes to the experimental task and domain,
     along with the concomitant changes to domain-specific vocabulary of
     interest.  It would be near-impossible for us to annotate data and train a
     neural agent on the new vocabulary of a study prior to its actual
     execution and the release of the data from it. 

\end{enumerate}

\section{Hypotheses}

We will test the following hypothesis: 

\begin{quote}
    \itshape
    Our rule-based event extraction system will be able to achieve a minimum
    precision of 91\% for simple labels and 75\% for complex labels, as defined
    below.
\end{quote}

\section{Approach}

The DialogAgent processes natural language texts from two sources: utterances
spoken by players, and text chats sent using the Minecraft chat
interface\footnote{For ASIST Study 3, participants are discouraged from using
    text chat, but it is not programmatically disallowed - for this reason, we
    still monitor it. Performer teams who use the outputs of the DialogAgent
    should inspect the \texttt{data.participant\_id} key in the DialogAgent
    output messages to make sure that it is not \texttt{Server}, which
correspnods to chat messages sent by the testbed infrastructure rather than
humans.}.

Our rule-based framework extracts \emph{events} from natural language. These
events can be categorized into two types:

\begin{enumerate}
    \item \emph{Simple events}: These events do not have arguments, and are
        found by string-pattern matching.
    \item \emph{Complex events}: These events take other events as arguments,
        and are found by a combination of pattern matching, dependency parsing,
        and POS\footnote{POS: Part of speech.} tagging.
\end{enumerate}
    
Each event is associated with a unique span of text and is assigned one or more
labels by the rule that extracts it. The labels we are using are organized into
an ontology.  If a rule assigns a label to an event from this ontology, the
parents of that label in the ontology are also assigned as labels for that
event.

\section{Evaluation}

We will conduct two kinds of evaluation for Study 3 - an F1 score evaluation on
a restricted set of event types, and a precision-only evaluation on the entire
label set.

\subsection{F1 score evaluation for subset of event types}

We will evaluate this capability by comparing the outputs of the DialogAgent to
events identified by human annotators. We will hire human annotators to
annotate a representative subset of utterances from Study 3. As the DialogAgent
currently extracts more than 100 event types, we have narrow down this
selection to 20 in order to avoid overwhelming our annotators. These events
will be selected for two qualities:

\begin{enumerate}

    \item \emph{Complexity}: Complex events rely on a dependency parse and are
        more unpredictable due to the fact that they rely on multiple
        conditions to trigger. For this evaluation, complex events are of
        greater interest to us, as their accuracy is harder to predict
        \footnote{We provide here a brief explanation as to why `simple' labels
            are less interesting. Consider a label for players talking about
            victims in the context of the Study 3 experimental task.  This
            label will be generated by a simple pattern matching rule that
            scans all tokens for a regular expression that looks something like
            this: \texttt{/(?i){\textsuperscript{$\wedge$} }victims?\$/} .
            This pattern will match any and all occurrences of `victim' (plural
            or singular). The precision and recall of this label now depends
            entirely on the quality of the transcription and on the way the
            players talk about `victims' (for example, they may use other terms
            (e.g., `this guy') as well). This makes this label much more
            predictable, than, say, one that relies on dependency graphs in
        addition to pattern matching.}.

    \item \emph{High frequency}: The selected label should occur with sufficiently
        high frequency in the dataset to generate a representative amount of data.
        As the events in our ontology are often highly domain-specific (and thus
        rare), we will circumvent this issue by letting annotators annotate for
        groupings of labels. For example, we have five different labels for players
        `needing something', such as a specific item, a specific role, or just help
        from other players. In order to lighten the load on our annotators, we will
        collapse these into one label.

\end{enumerate}

Annotators will receive transcripts of player communications. They will be
asked to annotate the transcripts for the 20 labels we have given them. With
this annotated data, we can write a script that will counter-check the
DialogAgent extractions against the annotations and calculate a representative
F1 score.

\subsection{Precision evaluation for the entire set of event types}

We will also run a separate evaluation for precision\footnote{For reasons of
economy, we restrict this evaluation to precision. Our expert team members can
judge produced labels for precision at a much higher speed than they can
annotate utterances for labels.}, done by team members and hired annotators who
are familiar with our DialogAgent labels.


\subsubsection{Results from pilot data} 

We have already done a preliminary precision evaluation of 1700 utterances from
Study 3 pilot data and have found that the simple events are extracted with
high precision.  In our preliminary evalution, simple labels yielded between
91-100\% precision.\footnote{This is for labels that had more than 20
occurrences in our evaluation.} `Complex' labels were found to be extracted
with 75-94\% precision\footnote{This is for labels that had more than 10
occurrences in our evaluation.}.

\subsection{Potential evaluation problems}

By subsuming multiple labels under one label for the general evaluation, we run
the risk of having no way of discerning if one label in a group is a
particularly challenging extraction. For example, consider the case where we
have five different labels corresponding to different types of questions, and
we subsume them under one generic `Question' label. After the annotations are
done, we calculate an F1 score for this new label. If the DialogAgent performs
weakly on one of the five labels (for example, an F1 score of 0.3), but the
others performed well, then the overall F1 score for this label will mask this
fact.

In order to combat this problem, we will run an evaluation for precision over
the entire label set, as described in the previous subsection. Unfortunately,
we will not be able to calculate recall for the entire label set, as this
process is simply too costly for the size of our label set (> 100).
