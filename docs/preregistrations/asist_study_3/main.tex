\documentclass[oneside,9pt]{memoir}

% Colors
\usepackage[dvipsnames]{xcolor}

% hyperref
\usepackage[%
    colorlinks=true,
    linktocpage=true,
    breaklinks=true,
    linkcolor=RoyalBlue,
    urlcolor=RoyalBlue,
    citecolor=RoyalBlue
]{hyperref}
\def\sectionautorefname{§}


% Fonts
\usepackage{fontspec}
\setmainfont[Numbers = {OldStyle, Proportional}]{TeX Gyre Pagella}

%\usepackage{amssymb}
\usepackage{unicode-math}
\setmathfont{TeX Gyre Pagella Math}

% Typography
\usepackage[tracking]{microtype}

% Images
\usepackage{graphicx}

% ======================================================================
% Memoir package - layout and styling
% ======================================================================

% The calc package is required for calculating readable text widths
\usepackage{calc}

\DeclareMathOperator*{\argmax}{arg\,max}

% Set outer and spine margins A wide right margin is chosen both for legibility
% of the typeblock and for tight integration of marginfigures and margin
% footnotes.

% Calculate widths in pts
\setlxvchars[\normalfont\normalsize] % about 66 characters per column
\setxlvchars[\normalfont\footnotesize] % about 45 characters per column

% Set left and right margins
\setlrmarginsandblock{1.15in}{3.5in}{*}
% Set upper and lower margins
\setulmarginsandblock{1.1in}{1.1in}{*}

% Set properties of margin notes, sidecaptioned floats, and footnotes in the
% margin.
\setmarginnotes{0.2in}{2.15in}{2\onelineskip}
\setsidecaps{0.2in}{2.15in}
\sidecapmargin{outer}
\renewcommand*{\sidecapstyle}{\normalfont\footnotesize}
\setsidecappos{c}


% Set footnotes in the margin rather than at the foot of the page
\footnotesinmargin
\setsidefeet{\marginparsep}{1.9in}{0.2in}{0pt}{\flushleftright\footnotesize}{*}

% Integrate the counters of the sidefootnotes and footnotes in margin.
\letcountercounter{sidefootnote}{footnote}
\setlength{\footmarkwidth}{0em}
\setlength{\footmarksep}{-\footmarkwidth}
\setlength{\footparindent}{1em}
\sideparmargin{outer}

\renewcommand*{\sideparfont}{\color{Maroon}\emphshape}
\renewcommand*{\sideparvshift}{2\baselineskip}
\marginparmargin{outer}

% Style the entries in the Table of Contents
\renewcommand*{\cftchapterfont}{\scshape\MakeTextLowercase}
\renewcommand*{\cftpartfont}{\color{Maroon}\scshape\MakeTextLowercase}
\captionstyle[\centering]{\footnotesize}
\captionnamefont{\footnotesize\color{Maroon}}

%% Bringhurst chapter and head styles with a Pedersen-type chapter number
\makechapterstyle{bringhurst}{%
	\renewcommand{\chapterheadstart}{} 
	\renewcommand{\printchaptername}{} 
	\renewcommand{\chapternamenum}{} 
	\setlength{\midchapskip}{15mm}
	\renewcommand*{\printchapternum}{%
        \begin{marginfigure}[0pt]
          \resizebox{!}{\midchapskip}{\color{Maroon}\emph{\thechapter}}
        \end{marginfigure}
      }
	\renewcommand{\afterchapternum}{} 
	\renewcommand{\printchaptertitle}[1]{%
	  \raggedright\Large\scshape\MakeLowercase{##1}}
	\renewcommand{\afterchaptertitle}{%
	  \vskip\onelineskip \hrule\vskip\onelineskip}
}
\setlength{\cftsubsectionindent}{0.6in}
\chapterstyle{bringhurst}
\headstyles{bringhurst}

\tightlists

% Headers and footers - page numbers, section headings, etc.
\makepagestyle{tufte}
\createmark{chapter}{left}{nonumber}{}{}
\makeoddhead{tufte}{}{}{\scshape\MakeTextLowercase{\leftmark}~~|~~\thepage}
\makeevenhead{tufte}{\thepage~~|~~\scshape\MakeTextLowercase{\rightmark}}{}{}
 \makerunningwidth{tufte}[8in]{8in}
\aliaspagestyle{chapter}{empty}
\nouppercaseheads
\pagestyle{tufte}

\checkandfixthelayout

% Bibliography management
\usepackage[%
    style=authoryear-comp,
    natbib
]{biblatex}
\addbibresource{bibliography.bib}

% ======================================================================
% Creating the title page
% ======================================================================

\begin{document}
\begin{center}
    {\LARGE ToMCAT Study 3 Preregistration}\\
    \bigskip
    \textbf{Authors}
\end{center}

\begin{tabular}{ll}
    \toprule
    Name & Institution \\\midrule
    Salena Ashton & University of Arizona\\
    Loren Champlin & University of Arizona\\
    John Culnan & University of Arizona\\
    Kobus Barnard & University of Arizona\\
    Emily Butler & University of Arizona\\
    Ruihong Huang & Texas A\&M University \\
    Meghavarshini Krishnaswamy & University of Arizona\\
    Clayton Morrison & University of Arizona\\
    Md Messal Monem & Texas A\&M University \\
    Remo Nitschke & University of Arizona\\
    Adarsh Pyarelal & University of Arizona\\
    Ayesha Qamar & Texas A\&M University \\
    Adam Ussishkin& University of Arizona\\
    Yuwei Wang & University of Arizona\\
    Andrew Wedel & University of Arizona\\
    Liang Zhang & University of Arizona\\
    \bottomrule
\end{tabular}

\bigskip

\tableofcontents* 

\chapter{Introduction}
\textbf{Adarsh Pyarelal}

\section{Purpose and structure}

This document serves the following purposes.

\begin{itemize}
    \item Declare the capabilities we aim to demonstrate for our online agent
        components in ASIST Study 3
    \item Declare in advance -- with as much specificity as we can muster --
        the offline analyses we plan to perform on ASIST study 3 data.
\end{itemize}


The goal of the preregistration process as originally devised is to separate
hypothesis-generating (exploratory) research from hypothesis-testing
(confirmatory) research. A large portion of the activities engaged in by TA1
performers in ASIST - e.g., development of models, algorithms, and systems -
does not fit neatly into the paradigm of hypothesis generation and testing.
For this reason, while we endeavor to specify our analyses and capabilities in
as much detail as we can, we do not specify specific social science style
hypotheses to be tested\footnote{Whether TA1 performers \emph{should} be
specifying formal, quantitative hypotheses is another discussion.}.

The primary motivation for the structure of this preregistration document
is to accelerate the writing up of manuscripts for publication by using the
structure provided by the preregistration process to plan ahead for
publications.

Each of the sections in this document is a `component preregistation' that
corresponds to a publication `seedling', with the content optimized for what we
call `copy-pasteability', that is, the ability to be copied verbatim into a
manuscript for publication with minimal changes. This is intended to reduce
duplicate effort between writing up preregistrations and publications.

Since we are a fairly large team, each section has the names of the primary
authors responsible for the content of the section displayed under the section
heading. Authorship is ascribed to those who have contributed substantially to
the ideation or writing of the content.

\section{Overview}

Broadly speaking, we are aiming to develop a suite of open-source technologies
for artificial social intelligence, with a focus on computational understanding
of spoken team dialogue. Each of the component preregistrations demonstrates a
capability that we believe is important for artificial social intelligence, and
is currently integrated or will be integrated in the near future into our ASI
agent that will be evaluated in ASIST study 3 or future ASIST experiments.

\begin{itemize}
    \item Describe the architecture of the ToMCAT ASI and ACs. Provide one or
        more architecture diagrams.
    \item Give a brief summary of the preregistrations and how they contribute
        to the overall architecture.
\end{itemize}

\begin{figure}
    \centering
    \includegraphics[width=6.5in]{images/nlp_architecture}
    \caption{Architecture of our multi-participant dialogue analysis system.}
\end{figure}

\chapter{Dialogue act classification}
\label{ch:da_classification}
\textbf{Ruihong Huang, Ayesha Qamar, Md Messal Monem Miah, Adarsh Pyarelal}

\section{Introduction}

Spoken natural language is the primary medium of communicating and coordinating
within the 3-person teams that will be participating in ASIST Study 3.  Given
this, AI agents will need to be capable of natural language understanding to
process the conversations to make informed and prompt decisions. Identifying
dialog acts (DAs) is one of the primary aspects of natural language
understanding. A dialog act can be identified as a method of defining the
semantic content and communicative function of a single utterance of dialog
\citep{Searle:1969}. Examples of dialog acts include `request', `question',
`acknowledgment', etc.
Dialog acts can provide important information about the user dialog turns and
set of possible system actions. The patterns of dialog acts
of different speakers could also potentially indicate the roles they play in a
team, e.g., leader, follower, etc., Thus, it is a necessary capability for our
ToMCAT architecture as well as for conversational agents in general.


Extensive research has been conducted on dialog act classification due to its
importance in natural language understanding. These research works have taken
different approaches in terms of datasets, machine learning models and input to
the models. The most widely used datasets are the Switchboard
Corpus (SwDA) and ICSI Meeting Recorder Dialog Act Corpus (MRDA)
\citep{Shriberg.ea:2004}. Most of the approaches use textual utterances from
dialog transcripts as input to their models. Initial efforts to identify DAs
used classic statistical machine learning models
such as support vector machines (SVMs) \citep{Henderson.ea:2012}, hidden Markov
models (HMMs) \citep{Stolcke.ea:2000}, and conditional random fields (CRFs)
\citep{Zimmermann:2009}.

More recently, deep learning models have been gaining in popularity for DA
classification. \citet{Liu.ea:2017} presented both CNN models and hierarchical
CNN+CNN\footnote{CNN: Convolutional neural network.} and CNN+RNN\footnote{RNN:
Recurrent neural network} models to classify dialog acts and showed that a
RNN/Bi-LSTM\footnote{LSTM: Long short-term memory (a type of neural network
architecture).} on top of a CNN model performs better than the other models
they considered.  \citet{Shen.ea:2016} showed that a neural attention model
with context information performed well on the SwDA dataset.
\citet{Raheja.ea:2019} achieved state-of-the-art results using a context-aware
self-attention model on the MRDA corpus. Another approach for DA classification
involves incorporating both lexical and acoustic features. This approach is
motivated by the fact that humans use vocalic features to express specific
dialog acts - for example, people often raise their pitch slightly at the end
of a question. \citet{Ortega.ea:2018} showed that their lexico-acoustic neural
network models outperformed similar models that took only lexical information
as input.

While these approaches provide excellent results on DA classification, they are
lacking in some important aspects.
First, most of these models require clean
transcripts as input. Achieving clean transcripts requires manual annotation,
which is both time consuming and costly. As a result, these models cannot be
used for DA classification in real time where the only transcripts available
are imperfect ones produced by automatic speech recognition (ASR) systems.
Second, the approaches do not explicitly address dialogs where mechanisms to
incorporate speaker identification and determining the discourse structure are
also crucial.
Finally, the aforementioned approaches often use only 5 types of high-level
tags for DA classification, which frequently do not entirely explain the
purpose of the DA under consideration. It is necessary to have both general
(high-level) and specific tags to truly understand a DA.

To address these limitations, we will design and implement a deep neural
network based DA classifier to process input utterances in real time. This will
allow downstream AI agents to use dialog act information in order to provide
timely interventions. We will test this capability offline on Study 3 data and
deploy it online during Study 4 data collection.

needs to understand the conversations of the player as the game progresses. For
the agent to be real time, we cannot rely on manual transcription. Instead we
have used a publicly available Automatic Speech Recognizer (ASR), named Google
Cloud Speech API to convert the utterances into text. This provides noisy text
in return which might harm the performance of the DA classifier. To compensate
for this, we will use acoustic features from the raw speech as well.

\section{Approach}
A Bi-LSTM based baseline model is already trained with clean transcripts. The
same model is trained again with the ASR generated transcripts and the
performance dropped significantly due to ASR noise and the highly overlapping
nature of the utterances in the meetings. To overcome this drop we will use a
fusion based audio-language model to leverage both lexical and acoustic
information of the utterances to successfully identify the DAs. In addition,
our approach will capture the threading structure within a dialogue that
involves detecting utterances falling within an adjacency pair (consisting of a
question and an answer utterance, or a request and an acceptance utterance) and
then linking them together. We will aim to eventually jointly learn both the
threading structure and DAs of a dialog in a multitask learning setting since
we expect the two tasks to benefit each other.

Currently we are using a multi-party dialog dataset MRDA
 that consists of 75 meetings each about an hour long,
where each utterance has one (out of 11) general and zero or more (out of 40)
specific tags. Once the experimentation is done, we will use a transfer
learning approach to train and test the model for study-2 data. 

MRDA data is a multiparty multilabel dialog act corpus. So, each utterance in
the dataset may have multiple tags associated with it which poses a problem for
building an efficient model that could correctly capture the dialog acts. One approach
to solve this multi-label problem is randomly choosing a label from 
the label combinations. However, this approach often does not reflect the most
prominent purpose of each utterance. So, we have drawn inspiration from label
disambiguation from related corpus and carefully looked at examples to come up
with a set of precedence rules for the labels and thus solved the multi-label
challenge. This approach significantly improves the performance of our baseline
model on MRDA data.

\section{Evaluation}

To evaluate our approach, we will use F1 score as the evaluation metric. For
multiclass classification problems, especially where the classes are highly
imbalanced, F1 score provides more insight than accuracy.  We also intend to
annotate ASIST data for dialog Acts so that the system trained on MRDA can be
fine-tuned on ASIST. The baseline model that we are currently using is a sequential
model consisting of a hierarchical LSTM. The macro F1 score for this model on MRDA
data is 40.77. However, when we evaluate the trained model on ASIST study 3 data, 
the accuracy and F1 score both degrade. The performance of the model on raw ASIST
data in terms of accuracy and macro F1 score is 30\% and 9.705 respectively. 
One of the key factors behind this performance degradation is noisy transcripts 
generated by the ASR. In order to solve this issue we have to manually correct the
transcriptions generated by the ASR. After correcting the transcripts, we have
observed a significant improvement of the model performance both in terms of accuracy
and F1 score. The accuracy and macro F1 score for the cleaned ASIST study 3 data are 
37\% and 18.86 respectively. Highlights of the results on some important classes are 
shown in the tables below. The first table is contains the result for noisy transcripts 
and the second table contains the results for corrected transcripts.

\begin{center}
\begin{tabular}{||c c c c||}
 \hline
 Label & Precision & Recall & F1 Score\\ [0.5ex]
 \hline\hline
 Yes/No Question & 0.80 & 0.39  & 0.52\\
 \hline
 Statement & 0.36 & 0.73 & 0.48\\
 \hline
 Command \& Suggestion & 0.52 & 0.46 & 0.49\\
 \hline
 Commitment & 0.33 & 0.01 & 0.01\\
 \hline
 Accept & 0.45 & 0.40 &  0.42 \\
 \hline
  Reject & 0.38 & 0.33 & 0.35\\
 \hline
  Understanding Check & 0.18 & 0.34 & 0.24\\
 \hline
\end{tabular}
\end{center}

\begin{center}
\begin{tabular}{||c c c c||}
 \hline
 Label & Precision & Recall & F1 Score\\ [0.5ex]
 \hline\hline
 Yes/No Question & 0.50 & 0.01 & 0.01\\
 \hline
 Statement & 0.32 & 0.73 & 0.44\\
 \hline
 Command \& Suggestion & 0.47 & 0.47 & 0.47\\
 \hline
 Commitment & 1.00 & 0.01 & 0.01\\
 \hline
 Accept & 0.39 & 0.31 &  0.34 \\
 \hline
  Reject & 0.50 & 0.35 & 0.41\\
 \hline
  Understanding Check & 0.18 & 0.34 & 0.24\\
 \hline
\end{tabular}
\end{center}

We can observe significant improvement for almost all of the interesting tags when the
utterances are corrected manually. Especially for the yes-no question class we can 
observe massive jumps in both precision and recall. However, some classes like commitment
are suffering from low recall performance even for the corrected transcripts. Moving forward
our next step will be to annotate enough data to fine-tune the model on ASIST to achieve
even better performance.

\chapter{Interdependence detection}
\textbf{Remo Nitschke, Yuwei Wang}
\section{Introduction}
\begin{itemize}
    \item Why is this a useful capability for an AI
        agent? How does it contribute to machine theory of mind/machine theory
        of teams?
        %% Not sure, maybe we need some inout here from the theory of mind people here?
        %% - Remo
    \item What are the existing state of the art approaches to this problem
        (cite relevant papers), and what are their limitations? 
        %% Same as above
    \item What is our approach, and how will it address those limitations?
        %% 
\end{itemize}

\section{Approach}
\begin{itemize}
    \item Provide details on our approach, including:
        \begin{itemize}
            \item What data is required for this work? If any of the required
                inputs are present in the table of Study 3 variables in the
                \href{https://docs.google.com/document/d/1GF7VsNF9R95IAaj6mVZUDV2mAX5ok1Bh6Tcm8zDpIkg/edit#heading=h.1ksv4uv}{TA3
                Study 3 preregistration document (table 3)}, please list those
                variables with the standardized verbose variable names in the table.
                %% V1 = MinecraftEntity_Event_Dialogue_Event_Dialogagent
        \end{itemize}
\end{itemize}

\section{Evaluation}
We will evaluate via manual evaluation by human annotators. We will hire human annotators to evaluate a representative chunk of utterances. Annotators will receive transcripts with Event Extractions available for each utterance. We then ask them to evaluate whether the present Event Extraction Labels are precise and whether any labels are \emph{missing} in the Event Extractions. This way we can calculate precision and recall for a chunk of data, allowing us to calculate a representative F1 score.
We will also run a seperate evaluation for precision,\footnote{For reasons of economy, we restrict this evaluation to precision. Our expert team-members can judge produced labels for precision at a much higher speed than they can annotate utterances for labels.} done by team members who are familiar with our dialog agent labels.
\subsection{Potential Evaluation Problems}
There are two potential issues we may face with this mode of evaluation. 

Annotators may be primed by presence of extraction labels. If an annotator is asked to decide whether utterance X qualifies for label Y, they are more likely to assing label Y if the dialog agent already has done so. We could potentially avoid this effect by seperating the tasks of annotation and evaluation. Annotators are asked to only annotate and the evaluation is then done automatically by comparing label output of the dialog agent to the label output of the annotator. Here we run the risk of clerical errors by annotators muddying the data. Since the argument structure of our dialog agent labels can be quite complex, we believe that our proposed course of action is the lesser evil.

At time of writing the dialog agent contains 149 event labels it can assign. A possible risk of hired annotators is that they simply cannot reliably evaluate that number of labels. Even with a provided annotated list of labels (containing small descriptions), we will have to assume that some false negatives\footnote{Due to the setup, we anticipate very few false positives and a larger amount of false negatives in the human evaluation. We assume that it is easier for an annotator to review whether a complex label is correct than to assign a complex label where none is given.} will occur. For this reason, we will run a seperate evaluation for precision only, done by expert members of the team who should have high familiarity with our labels.

\end{itemize}


\chapter{Sentiment analysis and personality trait detection}
\label{ch:sentiment_analysis}
\textbf{John Culnan, Adarsh Pyarelal}
\section{Introduction}

Personality trait detection, sentiment analysis, and related measures of
personal characteristics like emotion recognition are all frequently
used to characterize an object of study. As such, they provide an AI agent with
information about the background and states of individuals at any given time,
which may be used to allow the agent to interact with individuals in a manner
appropriate to their current state. Because understanding the personalities and
emotional states of other individuals is a key feature of collaboration among
people, it is a critical component of the identification of the overall mental
states of others as required for a machine theory of mind
\cite{Rabinowitz.ea:2018}. Providing an AI agent with this capability will
likewise help inform a machine theory of teamwork by allowing the AI agent to
incorporate a wider range of key information in performing its own role.

Our speech-based system of personal characteristic identification currently
includes predictions for emotion and dominant personality trait, and will be
extended during Study 3 to include both a sentiment task and an emotional intensity
task. The emotion task attempts to identify the emotion of each utterance from six
emotions (anger, disgust, fear, joy, sadness, surprise) and a neutral option. The
dominant personality trait task identifies which personality trait is dominant for
a person from among the Big Five personality traits (agreeableness, conscientiousness,
extroversion, neuroticism/emotional stability, and openness to experience). The
sentiment task seeks to identify whether the speaker shows positive, negative, or
neutral sentiment during an utterance. Finally, the emotional intensity task aims to
categorize emotional utterances as either showing a strong or weak level of emotion.

There are no existing state of the art (SOTA) models for the Study 3 dataset, as the
data collection has not been completed yet; however, SOTA models
exist for related datasets and tasks, some of which are incorporated into our
model as task-specific training data. For the Multimodal Emotion Lines Dataset (MELD)
\cite{Poria.ea:2019}, which consists of data from the television show Friends that
has been annotated for both sentiment and emotion, the
SOTA model is TODKAT \cite{Zhu.ea:2021}, which uses context-aware
text and audio transformer-based models for the emotion recognition task. The
First Impressions V2 dataset \cite{Ponce-Lopez.ea:2016}, consisting of the Big
Five personality trait classification for YouTube clips, was originally created
for use as a regression task of identifying the extent to which each speaker
has each personality trait; for the task of classifying the dominant personality
trait, the SOTA model is \citet{Culnan.ea:2021}. The Multimodal Opinion-Level
Sentiment Intensity (MOSI) dataset \cite{Zadeh.ea:2016} consists of single
speakers in YouTube clips providing opinions that are rated from highly
negative to highly positive. The current SOTA model for MOSI is
TupleInfoNCE \cite{Liu.ea:2021}, which uses contrastive learning on augmented
data.

One major limitation of current SOTA approaches is that the data on which they
are trained is not varied in source; typically, each task is trained
separately, so that when two or more tasks have similar or identical label
spaces, the model is retrained for each new task. The data within a given task
generally includes speakers with the same native language.  Multi-party data
furthermore frequently come from scripted sources, such as the television show
Friends \cite{Poria.ea:2019, Zhu.ea:2021}), and single party data from YouTube
\cite{Zadeh.ea:2016, Ponce-Lopez.ea:2016}). Finally, these SOTA models are
created around datasets that rely on accurate, hand-crafted transcriptions of
the audio data, without considering imperfections in the transcriptions. We
approach this challenge by examining the Study 3 data through multitask neural
networks that make use of both existing corpora as pre-training data as well as
the Study 3 data to perform the same tasks in a new domain. We further seek to
improve performance by giving special consideration to the imperfection of
automatic ASR transcriptions used with the Study 3 data.


\section{Approach}

Our approach includes a hard parameter sharing multitask model pre-trained on
tasks for sentiment \citep{Zadeh.ea:2016}), emotion \citep{Poria.ea:2019}),
personality trait \citep{Ponce-Lopez.ea:2016}),
and emotional intensity \citep{Livingstone.ea:2018}), then trained on data from
Study 3. Like \citet{Liu.ea:2021}, our model will make use of augmented data to
allow for more balanced classes in each dataset of interest.  Acoustic and text
data will be fed into the model separately, with text data masked to handle
noise created by automatic transcription. These text transcriptions will be
paired in an ensemble with predictions made by a trained wav2vec model
\citep{Schneider.ea:2019}, which will allow for further boosting of text
modality accuracy. Acoustic and text data will subsequently be concatenated
prior to making predictions about the classes for each utterance-level data
point. A schematic representing our approach is shown in
\autoref{fig:sentiment_model_schematics}.

\begin{figure}
    \begin{sidecaption}{%
        Schematic representation of our multitask model
    for classifying speech according to speaker's dominant personality trait,
    emotion, emotional intensity, and sentiment. The model makes use of both
    speech-based audio input and content-based text input. During Study 3,
    text representations will be augmented with output of a wav2vec system
    providing further information about the way the words were produced.
    Speech- and text-based modalities are fed through neural layers separately,
    then concatenated and fed through an additional neural layer before a final
    task-specific layer is used to make predictions on each of the four tasks
    simultaneously. }[fig:sentiment_model_schematics]
    \includegraphics[width=\textwidth]{images/sentiment_schematics_study3.png}
    \end{sidecaption}
\end{figure}

The experiments we propose to run with our model are based on an initial
analysis of two versions of our existing model's results on a subset of the
Study 3 Spiral 3 pilot data. The results of this initial analysis are shown in
Table \ref{tab:spiral3_analysis} In this initial analysis, we compared models
trained with and without information about the distribution of class labels in
the training (non-ASIST) datasets (i.e., with and without class weights).  For
the Study 3 Spiral 3 pilot data, the model trained with class weights achieved
an F1 score of 58.78 for emotion recognition (comparable to the score achieved
on the test partition of MELD). However, the model trained without class
weights performed better, reaching an F1 score of 72.64.

\begin{table}
    \small
    \centering
    \begin{tabular}{lcc}
        \hline
        Model type & Emotion & Personality Trait \\
        \hline
        Class weights & 58.78 & 21.58 \\
        No class weights & 72.64 & 37.65 \\
        \hline
    \end{tabular}
    \caption{Performance of trained models with and without class weights on
    emotion and dominant personality trait tasks. Numbers shown are F1 scores
    for emotion and personality trait. Gold labels for personality traits are
    determined using results of the Ten Item Personality Inventory survey,
    while gold labels for emotion were created through annotation of data
    points by two researchers.
    }
    \label{tab:spiral3_analysis}
\end{table}

For dominant personality trait identification, the results similarly showed
improvement when the model did not include class weights (F1 of 21.58 for Study
3 Spiral 3 data with class weights, F1 of 37.65 for this data without class
weights). These scores were calculated by using annotations for emotion and
sentiment by two researchers on our team as gold labels, as well as Ten Item
Personality Inventory (TIPI) survey results for personality trait
identification gold labels.

Results of both tasks may not have been representative of the true performance
of the model, as the Spiral 3 data used confederates (i.e., members of the
research team) rather than naive participants; it is possible these
participants did not respond to the TIPI in the same way that naive
participants would have, and this group of pilot participants may have been
calmer or more neutral in their speech due to their previous experience with
the missions and data collection procedure. To determine whether this is the
case, data from Study 3 will need to be annotated, with results from a portion
of Study 3 participants compared to the results of Spiral 3
participants\footnote{While data was collected from a set of naive participants
prior to the drafting of this preregistration, this data did not include the
necessary vocalic features for our analysis, due to technical issues. We expect
that the rest of the Study 3 data will contain these features.}.

Furthermore, while the model trained without use of information on the distribution
of gold labels in the training datasets outperformed the model trained with this
information, the model without distributional information did not make predictions
for all possible emotions and traits. In order to rectify this and attempt to
further improve the quality of model predictions, the distribution of data points
across all emotions and traits in the ASIST-specific data rather than pretraining
datasets will be incorporated into this model, as such information may provide an
additional boost to model performance.

The Study 3 data that will be incorporated into our models are:

\begin{enumerate}
    \item the
pre-experiment TIPI survey \\
(Participant\_CognitiveTenItemPersonalityInventory\_Survey\_PreExperiment)
    \item the
text output of the ASR system \\
(MinecraftEntity\_Observation\_Asr\_Speechanalyzer)
    \item the audio data and the
acoustic features extracted from this audio signal
(MinecraftEntity\_Observation\_Audio\_Speechanalyzer)
\end{enumerate}

In addition to this previously collected data, new
utterance-level annotations will be collected for sentiment, emotion, and
emotional intensity. The text output of the ASR system and the acoustic features
extracted from the audio signals will be used as input data in the model,
while the TIPI and new utterance-level annotations will be used as gold labels
to evaluate the performance of our model.

\section{Evaluation}

System evaluation will consist of examinations of the network's ability to make
correct predictions about sentiment, emotion, and personality traits using F1
scores calculated by comparing network predictions with survey responses and
human-annotated data as described above. To provide comparable scores to
existing models for these tasks, average F1 scores will be used, calculated as
the weighted average of the F1 score for each class in the task.  Bootstrap
resampling \cite{kohavi1995study} will be used to provide statistical
comparisons among models created for these tasks, enabling the identification
of the best model.

While F1 calculations provide information on the overall success rate of our
models, they are limited in their ability to identify areas of weakness. To
further evaluate the ability of each of our models to perform under the different
conditions available, a subset of incorrect data points will be examined.
Patterns of errors that demonstrate model weaknesses with a particular type of
data (such as difficulty in correctly identifying neutral sentiment in male
voices) will be used to inform further iterations of network updates.

\chapter{Entrainment detection}
\textbf{Meghavarshini Krishnaswamy, Andrew Wedel, Adam Ussishkin}
\section{Introduction}
    Entrainment (also referred to as `synchronization', `coordination', or `alignment') is the adaption of verbal and non-verbal actions by conversation partners to more closely resemble one another \parencite{borrie2014}. Its role in communication as been described as ``key for supporting important pragmatic aspects of conversation, including taking turns, interaction smoothness, building rapport, fostering social bonds, and maintaining interpersonal relationships''\parencite{borrie2019}. A time-sensitive cooperative task utilizing verbal communication would require participants to optimize their information channel. This makes entrainment a useful metric for assessing the degree of cooperation among team-mates.

    In speech, entrainment has been observed and analysed using rhythm and timing, pitch, vowel identity and acoustic features \parencite{reichel2018prosodic,borrie2019syncing}. Speech entrainment occurs in correlation with entrainment at other linguistic levels such as an increase in shared vocabulary and sentence structures \parencite{rahimi2017entrainment}. The focus of previous literature has mainly been on two-party conversations, while entrainment in multi-party conversations has mainly focussed on work-level and sentence-level entrainment. 

    Recent research on vocal entrainment has shifted from regression-based analysis to encoding-based neural networks for a few reasons: to model the non-linear relationship between vocalic features, to capture the complexity and diversity of both entrainment and disentrainment \parencite{nasir2020}. Further, most of research on entrainment uses manually processed transcriptions and gold labels for the extraction of vocalic features. 

    In order to expand the scale of research, it is important to examine the effect of ASR-based feature extraction on entrainment detection. Our approach will focus on assessing entrainment detection with ASR-based transcripts, and to see if the existing setup is conducive to observing entrainment as a function of duration. 

\section{Approach}
\begin{itemize}
    \item Data requirements: We will utilize the following data from the Study-3 input-
        \begin{itemize}               
               \item Audio recording
                Transcript
               \item Participant demographic information
               \item Self-evaluation
               \item MinecraftEntity\_Observation\_Asr\_Speechanalyzer
               \item MinecraftEntity\_Observation\_Audio\_Speechanalyzer
               \item MinecraftEntity\_Event\_Dialogue\_Event\_Dialogagent
        \end{itemize}
    \item Our objectives (in decreasing order of priority):
\begin{itemize} 
    \item Build a working statistical model for assessing entrainment while utilizing the vocalic features extracted by the ToMCAT-speechanalyzer system.         
    \item Train an encoding-based model using the methodology outlined in \textcite{nasir2020} for a simple classification task that identifies entrainment and directionality in a given conversation.
    \item Assess if the current Speechanalyzer system is set up to study entrainment trends as a time series. 
\end{itemize} 
\end{itemize}

\section{Evaluation}
Evaluation would be done along three lines:
\begin{itemize}

    \item Localized entrainment: Are there observable similarities between utterances by different speakers that occurred next to each other than utterances at different points in time?
    \item Global entrainment: after a given period of objective-oriented speech, are participants aligned more closely? 
    \item Entrainment along a time-series: is the current setup able to calculate trends in entrainment as a function of time?

\end{itemize}

\chapter{Online multi-agent plan recognition}
\label{ch:plan_recognition}
\textbf{Loren Champlin, Salena Ashton, Liang Zhang, Clayton Morrison}

\section{Introduction}

Plan recognition is the ability to understand and recognize logical structures
and patterns within a sequence of observed behavior. Developing this capability
for our AI agent would allow it to infer the latent plan structures,
strategies, and goals (i.e., `plan explanations') of not only individual human
agents but a team of agents as well. Producing these plan explanations
contributes to the logical belief structures that the AI agent must maintain as
part of its theory of mind and theory of teams
\citep{Tambe_1997,Baker_Tenenbaum_2014}. Furthermore, an AI agent could use
these inferred plan explanations to predict the teams' subsequent actions and
to help develop potential interventions for increasing team performance. Our AI
agent must recognize the conjoined plan explanations of multiple agents given
our team search and rescue setting and demonstrate this capability online
(i.e., as the team carries out their mission). While a highly sparse topic,
there are a few existing "state of the art" approaches for online Multi-Agent
Plan Recognition (MAPR). 

The most recent online MAPR approach by \citet{Argenta_Doyle_2017} uses an automated planner to produce feasible plan explanations by simulating potential sets of parameters, conditions, and plan structures needed to generate the observed behavior. Approaches of this type are known as \textit{plan recognition as planning approaches} \citep{Ramirez_Geffner_2009,Van-Horenbeke_Peer_2021}. The use of an automated planner requires constructing a symbolic representation of the problem domain in which MAPR is to be deployed, known as knowledge engineering. Plan recognition as planning approaches are typically highly expressive and are capable of solving plan recognition problems that involve high levels of logical reasoning. However, this high expressivity usually comes at the cost of the high computational complexity required by the automated planner \citep{Van-Horenbeke_Peer_2021}. \citet{Argenta_Doyle_2017} attempt to overcome this challenge by making several strict simplifying assumptions about their problem domain. Although, these assumptions reduce the computational complexity at the cost of expressivity (i.e., it limits the type of problems their approach can solve). Another limitation is that \citet{Argenta_Doyle_2017} only consider ``flat" knowledge representations, rendering their approach incapable of inferring more than just the end-goals that the agents are trying to achieve. However, human behavior tends to exhibit hierarchical structures or patterns such that simple actions are combined to produce more complex actions. In terms of having a complete theory of mind and theory of teams, an AI agent needs to understand how actions relate to each other at different levels of granularity and complexity, not just how they relate to the agents' end-goals. 

Our proposed method for online MAPR is also a \textit{plan recognition as planning} approach. Rather than make assumptions that limit the problems our approach can solve, we try to overcome the challenges of computational complexity by developing a highly efficient automated planning algorithm specialized towards doing online MAPR. Our automated planning algorithm combines the well-known Simple Hierarchical Ordered Planner (SHOP2) proposed by \citet{Nau_2003} and the Monte Carlos Tree Search (MCTS) single-agent plan recognition algorithm proposed by \citet{Kantharaju_Ontanon_Geib_2019}. We also draw heavy inspiration from known parsing algorithms for Probabilistic Context-Free Grammars (PCFG) \citep{Collins_2011}. As suggested by its partial adaption of the SHOP2 algorithm, our approach assumes that the agents' actions relate through a set of hierarchical structures or what is known as a hierarchical task network (HTN) \citep{Nau_2003,Russell_Norvig_2021}. As such, our approach produces the most likely task hierarchy, which is an instance of a HTN under specific initial conditions. These task hierarchies represent how actions relate to each other at different levels of granularity of complexity by showing how high-level actions (also known as compound tasks) can be decomposed into low-level actions. 

\section{Approach}
In a HTN domain representation, compound tasks are high-level actions that are composed of lower-level actions. These lower-level actions may be compound task themselves that need to be further decomposed or non-decomposable actions, which are sometimes referred to as primitive tasks or just actions \citep{Russell_Norvig_2021}. Depending on how abstractly a compound task is defined, there may be multiple sets of lower-level actions that could be combined to create the same compound task. These different sets of actions are known as ``methods", since they are different ways of decomposing the compound task \citep{Russell_Norvig_2021}. Additionally, methods typically have preconditions, which are specific conditions that must be satisfied by the current state of the problem domain for that method to be used for task decomposition. If a methods preconditions are satisfied, then that method is said to be applicable \citep{Russell_Norvig_2021}. Figure \ref{pr_fig:1} further illustrates the concepts of compound tasks, actions, methods, and task decomposition. In the illustration, both methods could applicable or only one of them could be applicable given the current state of the problem domain. In the former case, an automated planner must choose one of the methods for decomposition, leading to two different possible plans. Further details on HTN planning processes can be found in the automated planning literature. (e.g., the SHOP2 paper by \citet{Nau_2003}). 

\begin{figure}[h]
    \centering
    \includegraphics[width=1\textwidth]{images/htn_concepts}
    \caption{The compound task \textit{Travel to C} is shown to have two different methods for accomplishing the same task.} 
    \label{pr_fig:1}
\end{figure}

Our approach uses a HTN domain representation and an automated HTN planner to model the logical reasoning and decision-making process of a team of human agents. Compound tasks and methods are engineered in such a way as to represent the latent decisions that agents must make to complete their mission. From a generative perspective, these latent decisions are then what leads to the actions we observe from the human agents. With this concept in mind, we can have an automated planner generate a plan that matches the observed actions while recording the methods and task decompositions involved. As suggested in the introduction, this record is produced in the form of a task hierarchy which shows how compound task can be decomposed into the actions we observe. These task hierarchies are the plan explanations that give insight on the relationship between actions and the decision-making process of the agents. Figure \ref{pr_fig:2} illustrates the general concept of our online MAPR approach.

\begin{figure}[h]
    \centering
    \includegraphics[width=1\textwidth]{images/pr_as_planning}
    \caption{In this illustration compound task are represented as groups of upper-case letters and actions are singular lower-case letters. Multiple potential task hierarchies can be generated for the same observed plan.} 
    \label{pr_fig:2}
\end{figure}

As seen in figure \ref{pr_fig:2}, it is possible that there are multiple plan explanations for same observed actions. However, rather than obtain all potential plan explanations given an observed sequence of actions, our approach should be yielding the most likely plan explanation. With this in mind, our approach assigns a conditional probability $p(m | \bigcap_{n \in M_t} c_n(s), s)$ for a method $m \in M_t$ for a compound task $t$. $c_n(s)$ denotes a function that is 1 if $n \in M_t$ is applicable to $t$ for the current state $s$, and 0 otherwise. The conditional probabilities are then defined as,

\begin{equation} \label{pr_eq:1}
p(m | \bigcap_{n \in M_t} c_n(s), s) = \begin{cases} \alpha  & c_m(s) = 1 \\ 0 & c_m(s) = 0 \\ \end{cases}
\end{equation}

\begin{equation}  \label{pr_eq:2}
\sum_{m \in M_t} p(m | \bigcap_{n \in M_t} c_n(s), s) = 1
\end{equation}

The value denoted by $\alpha$ in (\ref{pr_eq:1}) of each conditional probability must be predefined. As part of our approach we have formulated a training algorithm for learning these conditional probabilities, however that training algorithm is not detailed here. 

As suggested, an automated planner uses the methods defined in a HTN domain representation to decompose a compound task or set of compound tasks into a plan. This derivation of a plan is an analogous process to deriving a sentence from a PCFG. A PCFG contains a vocabulary of both non-terminal symbols and terminal symbols, as well as a set of derivation rules for replacing non-terminal symbols with a sequence of both non-terminal and terminal symbols \citep{Collins_2011}. Given an initial non-terminal symbol and applying rules successively eventually yields a sequence of only terminal symbols (i.e., a sentence) \citep{Collins_2011}. Similar to the methods of our HTN domain representation, each derivation rule is assigned a probability. Thus, given a specific derivation of a sentence (i.e., a specific set of derivations rules used), the probability of that derivation is the product of the probabilities of each rule used \citep{Collins_2011}. Given the similarities, we define the probability of a specific task hierarchy for a given plan in the same way. We denote $m_t$ as the applicable method chosen to decompose task $t \in \tau_\pi$, where $\tau_\pi$ is a set of tasks representing some task hierarchy used to generate a plan $\pi$. Using (\ref{pr_eq:1}), we have that,

\begin{equation} \label{pr_eq:3}
p(\tau_\pi) = \prod_{t \in \tau_\pi} p(m_t | \bigcap_{n \in M_t} c_n(s_t), s_t) 
\end{equation}

$s_t$ is current state prior to the decomposition of task $t$. Since our approach is for online MAPR, we want our plan explanations to consist of a partial task hierarchy that shows how the agents' latent decisions generates their plans up to the current time step. We denote the agents' observed partial plan (i.e., their observed sequence of actions up to some current time step) as $\pi^*$. We then simply replace $\pi$ in (\ref{pr_eq:3}) with $\pi^*$ to get,

\begin{equation} \label{pr_eq:4}
p(\tau_{\pi^*}) = \prod_{t \in \tau_{\pi^*}} p(m_t | \bigcap_{n \in M_t} c_n(s_t), s_t) 
\end{equation}

Using (\ref{pr_eq:4}), the objective of our approach is then to compute 

\begin{equation} \label{pr_eq:5}
\hat{\tau}_{\pi^*} = \argmax_{\tau_{\pi^*} \in T_{\pi^*}} p(\tau_{\pi^*})
\end{equation}

In (\ref{pr_eq:5}), $T_{\pi^*}$ denotes the set of all partial task hierarchies that match the observed partial plan $\pi^*$.

SHOP2 uses a depth first search (DFS) algorithm to generate plans and as such we could modify it to generate $T_{\pi^*}$ and then compute $\hat{\tau}_{\pi^*}$ by comparing $p(\tau_{\pi^*})$ for all $\tau_{\pi^*} \in T_{\pi^*}$ \citep{Nau_2003}. This is however an extremely computationally complex procedure and not a feasible method, especially for online MAPR where we would need to run this same computation many times. In general, using any automated planning algorithm to generate $T_{\pi^*}$ is not feasible. 

We argue that reasonable method would be to instead sample partial task hierarchies from $T_{\pi^*}$ and estimate $\hat{\tau}_{\pi^*}$ instead, this estimate being denoted $\bar{\tau}_{\pi^*}$. \citet{Kantharaju_Ontanon_Geib_2019} had come to this same conclusion in their development of an approach to do single-agent plan recognition using a domain representation based on Combinatory Categorical Grammars (CCGs). As mentioned in the introduction, they use a MCTS algorithm to sample the space of plans and find a reasonable estimate of the most likely plan explanation for an observed plan, reducing the computational complexity of their approach significantly. Using a similar concept, we modified the SHOP2 algorithm to use a MCTS algorithm as opposed to DFS. Although we will not go into full details here, MCTS samples solutions from the search space, which in our case are partial plan hierarchies matching the agents' observed partial plans. The search algorithm uses each sample to compute statistics about the search space that inform it where ``good" areas of the search space are according to some utility function \citep{Browne_Powley_Whitehouse_Lucas_Cowling_Rohlfshagen_Tavener_Perez_Samothrakis_Colton_2012,Kantharaju_Ontanon_Geib_2019}.In our case, our utility function is $p(\tau_{\pi^*})$, which the algorithm attempts to maximize. MCTS balances its search between completely unexplored areas and areas of the search space that have yielded good results before \citep{Browne_Powley_Whitehouse_Lucas_Cowling_Rohlfshagen_Tavener_Perez_Samothrakis_Colton_2012,Kantharaju_Ontanon_Geib_2019}. Given enough search time, our MCTS algorithm will eventually compute $\hat{\tau}_{\pi^*}$, although the needed search time would be roughly equivalent to what SHOP2 would need to do the same task. Therefore we must limit the search time allowed and have our algorithm return the best $\tau_{\pi^*}$ it can find under a predefined search time, which is our estimate $\bar{\tau}_{\pi^*}$. There is a high chance that $\bar{\tau}_{\pi^*}$ may only be a local optima, but it is likely to be a sufficiently probable plan explanation given a reasonable amount of search time. 

In terms of the Study 3 data, our approach heavily utilizes the json mission event messages coming from the message bus during the Search and Rescue mission trials. These messages are primarily used as the observations for our plan recognition algorithm, but are also used along with the video of the trials as source material for knowledge engineering our HTN domain representation. We also use the messages in our training algorithm to learn the conditional probabilities as defined in \ref{pr_eq:1} and \ref{pr_eq:2}. The specific input variables that we use in our approach from the message bus are as followed, 

\begin{itemize}
\item MinecraftEntity\_Event\_Triage\_Simulator
\item MinecraftEntity\_Event\_Roleselected\_Simulator
\item MinecraftEntity\_Event\_Proximityvictiminteraction\_Simulator
\item MinecraftEntity\_Event\_Playerfrozenstatechange\_Simulator
\item MinecraftEntity\_Event\_Tooldepleted\_Simulator
\item MinecraftEntity\_Event\_Markerplaced\_Simulator
\item MinecraftEntity\_Event\_Markerremoved\_Simulator
\item MinecraftEntity\_Event\_Markerdestroyed\_Simulator
\item MinecraftEntity\_Event\_Victimpickedup\_Simulator
\item MinecraftEntity\_Event\_Victimplaced\_Simulator
\item MinecraftEntity\_Event\_Rubbleplaced\_Simulator
\item MinecraftEntity\_Event\_Rubbledestroyed\_Simulator
\item MinecraftEntity\_Event\_Victimnolongersafe\_Simulator
\item MinecraftEntity\_Event\_Missionstate\_Simulator
\item MinecraftEntity\_Event\_Location\_Locationmonitor
\item MinecraftEntity\_Event\_Victimsexpired\_Simulator
\item MinecraftEntity\_Observation\_State\_Simulator
\item MinecraftEntity\_Observation\_Fov\_Fovtool
\item MinecraftEntity\_Event\_Dialogue\_Event\_Dialogagent
\end{itemize}

\section{Evaluation}
Since it would be extremely difficult to objectively confirm a teams latent decision process used for a mission trial, we have devised an alternative evaluation method for testing the performance of our online MAPR approach. The algorithm described in our approach section, can be reconfigured to project forward in time past the agents' observed partial plan, effectively allowing it to predict their most probable next actions using $\bar{\tau}_{\pi^*}$ as a reference point for the agents' latent decision-making process. We reason that the closer $\bar{\tau}_{\pi^*}$ is to the true value, the more accurate the action prediction will be. Using this concept, we can indirectly measure how well our approach works by measuring how accurate the action predictions are given $\bar{\tau}_{\pi^*}$.

While our algorithm could technically project forward to the end of a team's mission trial, we expect the prediction accuracy to increasingly diminish as the gap between the end of the projected plan and the observed partial plan increases. Instead we evaluate our algorithms performance by having it use $\bar{\tau}_{\pi^*}$ to predict only the next few actions. Given an observed plan from a completed mission trial, we can simulate having a teams' observed partial plan at a set time points in their mission. At these set time points, we can run our online MAPR algorithm and then have our planner predict the next few actions, and compare these predicted actions against the teams' true actions. 

We will compute two different types of accuracy measures, the action allocation accuracy and the action sequence accuracy, which are accuracy measures used in other MAPR literature \citep{Kim_Chacha_Shah_2015}. The action allocation accuracy is the ratio of how many predicted actions were correct out of the number of actions predicted \citep{Kim_Chacha_Shah_2015}. The action sequence accuracy is computed by first dividing the correctly predicted actions into pairs and then counting how many pairs are in the correct order (regardless of what actions they are predicted to have between them). This count is divided by what the count would be if all pairs were correctly ordered \citep{Kim_Chacha_Shah_2015}. The average action allocation and average action sequence accuracy over all trials will be computed for each prediction point, as well as average accuracy measures over all prediction points over all trials. 

We will also measure the rate at which our prediction accuracies decrease as we increase the number of actions predicted. This can be done by picking a singular prediction point for each trial and measuring the action allocation accuracy and action sequence accuracy as we increase the number of actions predicted. Increases in te number of actions predicted will be done at set intervals (e.g., 2 actions, 4 actions, 6 actions, etc).  Average accuracies will then be computed for each interval over all trials. 

\chapter{Probabilistic modeling of team dynamics}
\label{ch:pgm}
\textbf{Paulo Soares, Kobus Barnard, Emily Butler}

\section{Introduction}
Human teams are interpersonal dynamic systems, where the individual team members
become subsumed by the collective whole in support of some shared goal or
function (Andrzej, K. Nowak et al., 2020; Cooke et al., 2013a; Eiler et al.,
2017; Fusaroli et al., 2014; Gorman et al., 2010; Wiltshire et al., 2017, 2019).
All higher-order team behavior, such as cooperation or competition, emerges from
coordination, which refers to causal interdependencies over time both within and
between the individual team members (Butner et al., 2014). At the individual
level, there are a number of sub-processes within each person, such as biology,
attention, cognition, emotion and motor behavior that mutually influence each
other to give rise to the person’s overall state at a given time (Butler, 2011,
2017). Similarly, at the interpersonal level, the individuals interact with each
other to give rise to the team’s overall state (Andrzej, K. Nowak et al., 2020;
Butler, 2011, 2017; Cooke et al., 2013b; Gorman et al., 2010; Letsky et al.,
2008).

One challenge for predicting team behavior and outcomes is that coordination
within the system, at both individual and collective levels, often results in
non-linear state changes over time. For example, a team may be unsuccessfully
trying to reach an agreement until one member reframes the problem in a way that
suddenly broadens everyone’s perspective, leading to immediate group consensus.
The non-linear change in the team’s state of agreement could be predicted by the
first person’s increased mental coordination in their mapping of the problem
(increased within-person coordination), which colloquially might be described as
“having the pieces fall into place,” followed by the team’s increased group
coordination as all members adopt the new perspective (increased between-person
coordination). Thus being able to recognize ongoing changes in coordination (in
this example in the domain of mental contents) could help an artificially
intelligent agent predict subsequent non-linear changes in team states and
performance outcomes. 

One important aspect of social coordination is that it is not mode- specific
(Eiler et al., 2017). Instead, coordination can be observed at every level of
the system, including neural, physiological, motor, perceptual, cognitive,
affective and linguistic levels simultaneously. It can also be observed at
multiple time-scales (Wiltshire et al., 2019). For example, during a
conversation, neural events transpire on the order of milliseconds, speech
production and gestures over seconds, and the conversation itself on the order
of minutes (Hasson et al., 2012). As such, observing coordination for the
purpose of prediction should be possible at any time resolution and any
modality, although different modes will change at different rates and thus must
be assessed at different time resolutions. For example, during a conversation
coordination between the electrical activity of people’s brains may be best
observed in milliseconds, while coordination of the semantic content of their
speech may be best observed at the level of speech turns.

Social coordination has been operationalized in many ways, often with approaches
that are mode or time-resolution specific. We prefer a more general approach,
since the same representation can be used to index coordination for any aspect
of the interpersonal system, from individual brain activity to collective
behavior sequences. One advantage of this for developing a socially intelligent
artificial agent is that the agent can learn about coordination from any
available observable information, which has important real-world consequences
for how useful such an agent will be, since it will not be constrained to
situations similar to those in which it is developed. The fundamental concept
that needs to be captured by a general representation of coordination is the
predictive interdependencies over time among the components of the system. See
the “Modeling” section below for the details of our representation.

We also prefer measures that are easily interpretable. Although coordination
should be observable from any measure of the interpersonal system, some measures
will map more closely to intuitive domains of coordination than others. For
example, coordination of speech behavior is theoretically observable from any
derived measure of vocal signals during a conversation. Along these lines, work
with natural language processing has made use of dozens of automatically
extracted features from speech for categorizing a broad range of targets, such
as personality types or emotional expression (CITE OpenSmile). Although
successful categorization has been achieved using various subsets of the
features, it is unclear what aspect of human behavior contributed to the
results, making it impossible to go from the results to theoretically based
interventions. For the present research, therefore, we focus on coordination of
two vocal features that map directly to pitch and loudness of human speech (see
Methods for details), both of which could be targeted by an intervention as
simple as asking a person to speak softly. 

Finally, we also prefer models that are predictive and interpretable. XXXXXXX

\section{Hypotheses}
The proposed research will use data from ASIST Study 3, to be conducted in the Spring of 2022. Participants will take part in 3-person teams and will play a game where they attempt to save virtual victims in a Mindcraft Search and Rescue scenario. We will test the following hypotheses:

1) Including coordination as a node in a probabilistic graphical model
\textbf{will improve} prediction of team process and effect measures (see Measures for details)  as compared to a comparable model that does not include coordination.

2) Teams with higher coordination will have higher scores on measures of team process and effects (see Measures for details) than teams with lower coordination.

3) The agent has a set of three interventions to encourage oral communication among
team members. Teams where the agent intervenes will have higher coordination,
and higher scores on measures of team process and effects (see Measures for
details), than teams where the agent does not do this. For teams selected for
intervention will get interventions contingent on activities and also on how
many interventions have been done so far to avoid intervening too much.  The
interventions are:
\begin{enumerate}
    \item
    If a team member places a marker block (a form of communication provided in
    the game) but does not say they did, then the agent will suggest that they
    tell their team members that they placed it. 
    \item
    For some teams, if a team member needs help (to stabilize a victim or exit a
    blocked room) but does not say they do, then the agent will suggest that
    they tell their team members that they need assistance. 
    \item
    For some teams, if a team member asked for help but did not get a reply from
    any of the other team members, then the agent will suggest that they tell
    the others again that help is needed. 
\end{enumerate}

\section{Approach}

\subsection{Design}
Our analyses to test Hypotheses 1 and 2 will make use of data from all of the
participants in the study (e.g., all possible between- and within- participant
conditions). As such, testing whether including coordination in the model
improves prediction of team process and effects measures (Hypothesis 1) and
testing the association between coordination and team process and effects
measures (Hypothesis 2) will each be based on 224 missions (8 conditions X 14
teams X 2 missions). Our analyses to test Hypothesis 3 will make use of data
from the ``No-Advisor” and ``UAZ-Agent” conditions, and so testing the effect of
our intervention (Hypothesis 3) will be based on 28 ``No-Advisor” control
missions (14 teams X 2 missions) compared to 28 ``UAZ-agent” intervention
missions (14 teams X 2 missions).

\subsection{Measures}
\textbf{Coordination indicators}. As discussed above, team coordination is
theoretically observable across modalities and time resolutions for all of the
components of the interpersonal system. Given the constraints of what will be
measured for ASIST Study 3, we will focus on two vocalic measures that are to be
provided in real-time (every 10ms) during the missions as indicators for
modeling linguistic coordination. The first, ``F0final\_sma" is the smoothed
moving average of fundamental frequency. It corresponds to a speaker’s pitch.
The second, ``wav\_RMSenergy\_sma" is the Root Mean Square energy measure, which
provides an estimate of the intensity of the sound, which correlates with what
humans perceive as loudness. 

\textbf{Team Process and Effects Measures}. We will include 5 team process
measures and 2 team effects measures, each taken from the full set of measures
that will be available for ASIST Study 3. This subset was chosen due to either:
1) having theoretical connections to coordination (i.e., synchronization,
coordinative communications, belief differences) or 2) involving the use of
markers for communication, which is the direct target of our intervention
(threat room and victim type communication), or 3) representing a global measure
of team effectiveness (score and error rate).

The team process outcome measures are:
\begin{itemize}
    \item
    ASI-M2 Synchronization (e.g., inter-subtask latency); \textbf{Measure}: Mean latency
    between subtasks of teamwork tasks per trial. Specifically, the mean latency
    between victim discovery and victim rescue computed as the average over all
    correctly rescued victims (time of delivery of victim to correct rescue site
    - time victim was discovered). Where time victim was discovered = the
    earliest (or minimum) of (1) Transporter signalling device for the room
    containing the victim, or (2) Medic completes stabilization of the victim.
 
    \item
    ASI-M5 Coordinative Communications; \textbf{Measure}: A quantitative characterization
    of the complexity of team communication using entropy in the dynamical
    system. Entropy is defined as the level of irregularity in the flow of
    communications between team members.(i.e., Team Communication Dynamics).
    Typical components of the measure are: (1) Intensity and distributions of
    team communications are calculated using calculations of over message
    frequency, duration, and intra-team variance. (2) Temporal patterns of team
    communications are captured using timestamped communication events and
    analyzed with recurrence quantification analysis.

    \item
    Rutgers Threat Room Communication AC; \textbf{Measure}: \% of threat rooms that had a
    threat room marker block outside its entrance at some point during the
    mission.  

    \item
    Rutgers Victim Type Communication AC; \textbf{Measure}: count of victim
    type marker blocks placed near the correct type of A or B victims.
    
    \item
    Rutgers Belief Difference AC; \textbf{Measure}: entropy of beliefs about victim
    distribution by comparing three models in which (1) players share
    information perfectly, (2) players don't share any information, and (3)
    players only share information via marker blocks.  
\end{itemize}

The team effects measures are: 

\begin{itemize}
    \item
    ASI-M1 ASI Utility (score) 
    \item

    ASI-M3 Error Rate; \textbf{Measure}: Percentage of
    subtasks of teamwork tasks that are completed per trial, measured as below.
    Specifically, the percentage of discovered victims who are not evacuated.
    Measure = 1 - (\# victims rescued / \# victims discovered)
\end{itemize}

\subsection{Modeling}

\subsection{Intervention}

\section{Evaluation}




% hypotheses?
% - TMM model can predict behavior
% - TMM agent interventions improve performance
% - Closed loop communication is predictive of team performance
\printbibliography
\end{document}
