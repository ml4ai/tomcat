\documentclass[oneside,9pt]{memoir}

% Colors
\usepackage[dvipsnames]{xcolor}

% hyperref
\usepackage[%
    colorlinks=true,
    linktocpage=true,
    breaklinks=true,
    linkcolor=RoyalBlue,
    urlcolor=RoyalBlue
]{hyperref}
\def\sectionautorefname{§}


% Fonts
\usepackage[osf]{newpxtext}
\usepackage{newpxmath}

% Typography
\usepackage[tracking]{microtype}

% Images
\usepackage{graphicx}

% ======================================================================
% Memoir package - layout and styling
% ======================================================================

% The calc package is required for calculating readable text widths
\usepackage{calc}

% Set outer and spine margins A wide right margin is chosen both for legibility
% of the typeblock and for tight integration of marginfigures and margin
% footnotes.

% Calculate widths in pts
\setlxvchars[\normalfont\normalsize] % about 66 characters per column
\setxlvchars[\normalfont\footnotesize] % about 45 characters per column

% Set left and right margins
\setlrmarginsandblock{1.15in}{3.5in}{*}
% Set upper and lower margins
\setulmarginsandblock{1.1in}{1.1in}{*}

%% Bringhurst chapter and head styles with a Pedersen-type chapter number
\makechapterstyle{bringhurst}{%
	\renewcommand{\chapterheadstart}{} 
	\renewcommand{\printchaptername}{} 
	\renewcommand{\chapternamenum}{} 
	\setlength{\midchapskip}{15mm}
	\renewcommand*{\printchapternum}{%
        \begin{marginfigure}[0pt]
          \resizebox{!}{\midchapskip}{\color{Maroon}\emph{\thechapter}}
        \end{marginfigure}
      }
	\renewcommand{\afterchapternum}{} 
	\renewcommand{\printchaptertitle}[1]{%
	  \raggedright\Large\scshape\MakeLowercase{##1}}
	\renewcommand{\afterchaptertitle}{%
	  \vskip\onelineskip \hrule\vskip\onelineskip}
}
\setlength{\cftsubsectionindent}{0.6in}
\chapterstyle{bringhurst}
\headstyles{bringhurst}

\checkandfixthelayout

% Bibliography management
\usepackage{biblatex}
\addbibresource{bibliography.bib}

% ======================================================================
% Creating the title page
% ======================================================================

\title{ToMCAT Study 3 Preregistration}
\date{}

\begin{document}
\maketitle
\tableofcontents* 

\chapter{Introduction}
\textbf{Adarsh Pyarelal}
\section{Structure}
\section{Author contributions}
% Talk about structure of preregistration document, purpose (paper seedlings),
% author contributions.

\chapter{Dialogue act classification}
\textbf{Ruihong Huang}
\section{Introduction}

\begin{itemize}
    \item Why is dialogue act classification a useful capability for an AI
        agent? How does it contribute to machine theory of mind/machine theory
        of teams?
    \item What are the existing state of the art approaches to this problem?
        Cite relevant papers.
\end{itemize}

\section{Approach}
\begin{itemize}
    \item What is our approach to dialogue act classification in study 3?
        Provide details including:
        \begin{itemize}
            \item What data is required for this work? If any of the required
                inputs are present in the table of Study 3 variables in the
                \href{https://docs.google.com/document/d/1GF7VsNF9R95IAaj6mVZUDV2mAX5ok1Bh6Tcm8zDpIkg/edit#heading=h.1ksv4uv}{TA3
                Study 3 preregistration document (table 3)}, please list those
                variables with the standardized verbose variable names in the table.
        \end{itemize}
    \item Why do we believe it will improve upon the state of the art?
\end{itemize}

\section{Evaluation}

\chapter{Interdependence detection}
\section{Introduction}
\section{Approach}
\section{Evaluation}

\chapter{Multimodal sentiment analysis}
\textbf{John Culnan}
\section{Introduction}
Previous work: \cite{culnan-etal-2021-ire}
\section{Approach}
\section{Evaluation}

\chapter{Online entrainment detection}
\textbf{Meghavarshini Krishnaswamy, Andrew Wedel, Adam Ussishkin}
\section{Introduction}
\section{Approach}
\section{Evaluation}

\chapter{Online multi-agent plan recognition}
\textbf{Loren Champlin}
\section{Introduction}
\section{Approach}
\section{Evaluation}

\chapter{Probabilistic modeling of team dynamics}
\textbf{Paulo Soares, Kobus Barnard, Emily Butler}
\section{Introduction}
\section{Approach}
\subsection{Model}
\section{Evaluation}

% hypotheses
% - TMM model can predict behavior
% - TMM agent interventions improve performance
% - Closed loop communication is predictive of team performance
\printbibliography
\end{document}
