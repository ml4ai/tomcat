\section{Task 2: Closed loop communication (CLC) detector}

Closed-loop communication (CLC) is often recommended in the team research
literature as a communication behavior that can guarantee the accuracy of
information exchange. In this project, we developed a rule-based algorithm to
automatically detect closed-loop communication in teams communicating via
natural language.
We built upon existing technologies from the ToMCAT project, including modules
for automatic speech recognition (ASR) and event extraction from natural language in real
time.

\subsection{Milestones}

We were able to meet the first milestone for this task, however, we did so
using data from ASIST Study 3
\cite{huang_freeman_cooke_colonna-romano_wood_buchanan_caufman_2022} rather
than ASIST Study 2
\cite{huang_freeman_cooke_dubrow_colonna-romano_wood_buchanan_caufman_2021}, as
this data was of higher quality. Specifically, the quality of the ASR
transcriptions was much higher due to the switch to a different Google Cloud
Speech model that is able to handle background noise much better, along with
speech context hints in the forms of commonly occurring phrases in the domain.
In this experiment, teams of three humans collaborate to save victims of an
office building collapse in an urban search and rescue mission simulated in
Minecraft.

We also preregistered our closed-loop communication detection approach on OSF
\cite{pyarelal_2022} prior to the collection of ASIST Study 3 data, to support
greater transparency and open science.

However, we were not able to meet the second milestone for this task, as it
depended on the development of the WASD (\autoref{sec:wasd}), which we were not able
to complete in time to collect pilot data with two humans.

\subsection{Approach}

There are three distinct phases in a complete Closed-loop communication:

\begin{enumerate}
    \item Call-out: The sender initiates a message.
    \item Check-back: The receiver acknowledges the message, usually by
        paraphrasing or repeating the main information of the message.
    \item Closed-loop: The sender verifies that the message has been received
        and interpreted correctly.
\end{enumerate}

Our rule-based event extraction system leverages the Odin framework
\cite{valenzuela-escarcega-etal-2016-odins}. It currently contains 420 active
rules that can extract from simple to complex events, as well as entities and
sentiment. Based on these event rules, we develop an algorithm to detect the
three phases of CLC:

\begin{enumerate}

    \item First, the Call-out phase will be detected by rules in the
        DialogAgent.  Labels including HelpRequest, Instruction, and NeedAction
        are used as triggers for this event.

    \item Next, we examine a window of the five subsequent utterances in the
        dialog. If we find Move or Commitment labels by other team members as
        well as the same semantic labels as in the Call-out message, then this
        indicates there is a Check-back phase. However, we also care about what
        we term `weak' Check-back situations, where the team member
        acknowledges the Call-out but does repeat the main information of the
        Call-out message. We assign the weak version of Check-back a 0.5 score,
        compared to the full score of 1. 

    \item Finally, within five utterances window, if we see the sender verified
        the information of the Check-back with the Agreement label, we can
        determine that this is a closed-loop communication. The final score of
        the Closed-loop communication detected is the sum of the scores of the
        three phases.

\end{enumerate}

\subsection{Evaluation}

To evaluate our algorithm, we annotated 3239 of the utterances from ASIST Study
3 dialog transcriptions and compared the outputs of our algorithm against the
annotations to compute precision, recall, and $F_1$ scores.


\subsection{Results}

Our CLC detection system extracted 978 closed-loop communication events,
with scores ranging from 1.0, corresponding to events in which only the
Call-out is detected but not the Check-back
and Closed-loop, to 4.0, for which two team members had checked back to one
call-out, and the initiator closed the loop at the end. However, our annotators
only detected 155 closed-loop communication patterns, indicating that the
rule-based CLC detection system in its current form suffers from a large rate
of false positives. We are exploring ways to mitigate this issue, including
making the conditions for CLC detection more stringent.
A \emph{t}-test between the CLC scores marked by the annotators and the
CLC scores assigned by our detection system shows that the CLC detection system
does not give significantly higher scores for CLC detection than the annotators
($p > 1$). 

\autoref{tab:clc} shows the results of our evaluation. Because the participants
in our game tasks are not specifically trained to use Closed-loop
communication, CLC events are relatively rare in our data. We can
see from the table that although the overall $F_1$ is as high as 0.79, the score
is largely boosted by the `NA' labels, which correspond to utterances that are
not recognized by the annotators and our system as part of CLC events.
However, for the individual phases of the CLC, we do see higher
recall scores than precision. This indicates that our rule-based CLC detection
system detects CLC events fairly reliably when they are present, but also
labels a fairly large number of utterances as part of CLC events when they are
not.

\subsection{Error Analysis}

We identified three major reasons for the low performance of our initial
rule-based CLC detection system.

\begin{enumerate}

    \item Firstly, for the Call-out phase,
there is a competing nature between precision and recall.

    \item For example, the Call-out is usually a question initiated by the
        initiator, but there are also a fair amount of questions that do not
        necessarily initiate a CLC session, such as ``shall we go'', and
        ``where are you''.

    \item Secondly, when we look for recursive labels between Call-out and
        potential Check-backs, overgeneralization could easily happen to
        introduce noise. A lot of the time, the recursive labels are just very
        common labels like Room or Victim, but the speakers are actually
        talking about different things. We would need to increase the
        strictness of the rules to account for this.

    \item Finally, the overmatching of the Close-loop phase is generally caused
        by only looking for a simple `Acknowledgment' or `Gratitude' label.
        However, these are also very common labels in the data, which led to
        overmatching.

\end{enumerate}



\begin{table}
    \centering
    \begin{tabular}{lrrrr}
        \toprule
                   & Precision & Recall & $F_1$ & Support\\\midrule
        NA         & 0.985     & 0.722  & 0.833 & 3084\\
        Call-out   & 0.123     & 0.774  & 0.212 & 155 \\
        Check-back & 0.118     & 0.721  & 0.203 & 147\\
        Close-loop & 0.021     & 0.550  & 0.041 & 20\\
        Overall    & 0.949     & 0.702  & 0.798 & 3239\\
        \bottomrule
    \end{tabular}
    \caption{Results of the evaluation of our rule-based CLC detection system}
    \label{tab:clc}
\end{table}

\subsection{Conclusion and future plans}

In this project, we explored a novel method to automatically detect closed loop
communication in spoken dialog. However, the performance of our algorithm is
not as high as we would have hoped. Specifically, the precision is too low to
inspire trust for downstream applications. We see many areas that we can work
on to improve the current system.  Besides tightening the rules, we are also
planning to explore machine learning approaches to deal with some of the issues
with the rule-based system. The annotations we performed for this task can
serve as valuable training data for a machine learning classifier.
