\section{Dashboard}

The Tomcat-Dashboard is a cross-platform user interface for controlling and visualizing scientific experiments.

\subsection{Installing dependencies}

The following software components are  dependencies required to build and run the dashboard.

\begin{enumerate}
    \item cmake: Build software used for generating cross-platform Makefiles
    \item Mosquitto: MQTT message bus broker
    \item paho.mqtt.cpp: Client library for MQTT message bus integration
    \item nlohmann-json:  Library for creating and parsing JSON messages
    \item wxWidgets: Cross-platform application and GUI development library
    \item Boost: Multipurpose library
\end{enumerate}

\subsubsection{macOS}

On macOS, all dependencies can be installed using the MacPorts package manager.

\begin{lstlisting}
port install cmake mosquitto paho.mqtt.cpp nlohmann-json wxWidgets-3.2 boost
\end{lstlisting}

\subsubsection{Ubuntu}

Similarly, from Ubuntu 21.10 onwards, all dependencies can be installed through
the apt-get package manager.

\begin{lstlisting}
apt-get install cmake mosquitto
apt-get install libpaho-mqtt-dev libpaho-mqttpp-dev libssl-dev
apt-get install nlohmann-json3-dev libboost-all-dev
apt-get install libwxgtk3.0-gtk3-dev
\end{lstlisting}

Note: The paho.mqtt.cpp library  is not available through  apt-get before
Ubuntu 21.10. If using a prior version of Ubuntu, the library will need to be
built from source.

Building

The Tomcat-Dashboard uses CMake to generate cross-platform Makefiles. This
makes it easy to build the program once all dependencies are installed. After
cloning the repository, to build the dashboard, run the following from the root
directory.

\begin{lstlisting}
mkdir build && cd build
cmake ..
make
\end{lstlisting}


Running
Launch MQTT broker
The Tomcat-Dashboard connects to an MQTT message bus for processing incoming data and communication between components. Before running the dashboard, an MQTT broker must be launched.
MacOS
mosquitto -p 1883

Ubuntu
service mosquitto start
Configure MQTT broker
By default, the dashboard will attempt to connect to an MQTT broker running on
port 1883 of localhost. This can be configured in the conf/app.json file by
modifying the |mqtt\_host| and mqtt\_port fields.

mqtt_host: (default=0.0.0.0) The host that the MQTT broker is running on
mqtt_port: (default=1883) The port that the MQTT broker is running on
Run dashboard

\begin{lstlisting}
./gui
\end{lstlisting}



Visualization
ASR
Example ASR Message:
{
  "data": {
    "text": "I am going to save a green victim.",
    "is_final": true,
    "id": "59678a5f-9c5b-451f-8506-04bc020f2cf3",
    "participant_id": "participant_1",
    "start_timestamp": "2021-01-19T23:27:57.978016Z",
    "end_timestamp" : "2021-01-19T23:27:58.633076Z",

   },
  "header": {
    "timestamp": "2021-01-19T23:27:58.633076Z",
    "message_type": "observation",
    "version": "0.1"
  },
  "msg": {
    "timestamp": "2021-01-19T23:27:58.633967Z",
    "experiment_id": "e2a3cb96-5f2f-11eb-8971-18810ee8274e",
    "trial_id": "256d1b4a-d81d-465d-8ef0-2162ff96e204",
    "version": "3.3.2",
    "source": "speech_analyzer_agent",
    "sub_type": "asr:transcription"
  }
}


Usage
To visualize incoming ASR utterances, navigate to the “ASR” tab in the menu sidebar. At the start of the trial, the right-side panel will display the text “Waiting for ASR messages…”.
Image 1: ASR visualization at trial start

As the trial progresses, and ASR messages come in, they will fill the panel one line at a time. In addition to the utterance text itself, each line will also contain a timestamp of when the utterance was generated, and a participant id colored to match the associated color of the participant.
Image 2: ASR visualization during ongoing trial
This color can be Red, Blue, or Green, and the mapping between color and participant comes from the Trial Start message.
Charting
Chart Configuration
The data being charted can be configured by modifying the “ChartWidget” section in conifg/conf.json.

"ChartWidget":{
             	"topics":[
                    	"score"
            	],
            	"x_axis_field": "Time",
            	"x_axis_label": "Time",
            	"y_axis_field": "Score",
            	"y_axis_label": "Score",
            	"panel_name": "CHART_PANEL"
 }

The dashboard will listen for incoming messages on any topics listed in the “topics” field. When processing messages, it will then check for the fields defined by “x_axis_field” and “y_axis_field”, create an x/y point, and add that point to the chart. “x_axis_label” and “y_axis_label” are used to add a label to the x-axis and y_axis of the chart. Finally, “panel_name” is used internally by the dashboard, and should not be modified.
Example Chart Message
{
    	"data":{
            	"Time": 0.2,
            	"Score": 10
    	}
}



Note:
1. The “Time” and “Score” fields are not at the root level of the JSON message, but a part of the “data” field. This is to keep the formatting consistent with for other message types used by the dashboard.
2. Each message should contain exactly one x/y pair. Other fields can be included, but will be ignored
3. Data points should be either an integer or float value

Usage
To visualize data being charted, navigate to the “Chart” tab in the menu sidebar. At the start of the trial, the right-side panel will display an empty graph. If a label for the x-axis or y-axis was defined in the configuration, those labels will be displayed on the axes.  By default, the graph will be centered around point (0,0)
Image 3: Chart visualization before trial start

When adding a data point, if the data point is outside the visible range, the chart will automatically fit the data and adjust the x-axis and y-axis so that the new data point can be seen.


Image 4: Chart visualization during ongoing trial
