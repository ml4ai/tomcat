\chapter{Summary}

In this report we describe the activities performed and milestones met for the
following SensorLab seed grants:

\begin{enumerate}
    \item Automated real-time detection of closed-loop communication in spoken
        dialogue (SensorLab Student Seed Grant).
    \item Development of an open-source dashboard for team communication
        experiments (SensorLab Seed Grant).
\end{enumerate}

The projects were designed to be complementary to each other, and intended to
enable studies of real-time spoken team communication in ambulatory settings.
See \autoref{fig:architecture} for an overview of the proposed architecture.

\begin{figure}
    \includegraphics[width=\textwidth]{figures/architecture.pdf}
    \caption{%
        Proposed architecture for the SensorLab Student Seed Grant
        project. The dashboard is not pictured in this architecture, but would be
        integrated into it.
    }
    \label{fig:architecture}
\end{figure}


\paragraph{Progress on milestones} The scope of our proposed tasks was
ambitious, and we were unfortunately not able to finish the development of the
wearable audio streaming device (WASD) (\autoref{sec:wasd}).  However, we are
happy to report that we were able to (i) develop and evaluate a rule-based
closed-loop communication detection algorithm (\autoref{sec:clc}) and (ii)
develop a prototype of the proposed dashboard for visualizing and controlling
team communication experiments (\autoref{ch:dashboard}).

\paragraph{Training} The projects have contributed to training and skill
development for 3 undergraduates, 3 master's students, and a research
programmer.

\paragraph{Software} The project have resulted in open-source codebases for the
software components, listed below.

\begin{itemize}

    \item The code for the dashboard is available at
        \url{https://github.com/ml4ai/tomcat-dashboard}.
    \item The code for the audioStreamer program is available at
        \url{https://github.com/ml4ai/tomcat/tree/master/exe/audioStreamer}.

\end{itemize}

The code for the closed loop communication detection algorithm is not yet up on
a public repository, but will be put up eventually. We are also happy to share
it sooner upon request.

\section{Current status and future work}

\paragraph{Extramural funding} In order to develop a competitive external NSF
grant proposal involving the SensorLab, further work needs to be done to
complete the development of the WASD. We plan to continue working with the
Raspberry Pi devices purchased with this seed grant in order to finish
developing and testing the audio streaming pipeline.

Simultaneously, we will improve our closed-loop communication detection system,
exploring both refining the existing rule-based system, as well as a machine
learning approach.

Once pilot data collection starts in earnest, the dashboard will prove a
valuable tool for experiment control, visualization, and troubleshooting.
