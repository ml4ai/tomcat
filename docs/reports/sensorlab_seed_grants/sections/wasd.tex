\section{Task 1: Wearable audio streaming device (WASD)}
\label{sec:wasd}

The goal of this task was to develop a wearable audio streaming device to
study spoken team dialog while bypassing the problem of audio source
separation.

Specifically, we wanted to write a program that would capture audio chunks on a
Raspberry Pi Zero 2 W device using a USB microphone, and stream the chunks to a
server over websockets. The chunks would then be fed to an automatic speech
recognition system to produce real-time transcriptions that could then be used
for further downstream analysis.

\subsection{Accomplishments and lessons learned}

Unfortunately, while made a good deal of progress on this task, we were not
able to fully meet our proposed milestones. However, we did accomplish the
following:


\begin{itemize}

    \item \textbf{Hardware expertise development.} We were able to set up and
        get familiar with the Raspberry Pi devices, and setting up networking
        on Ubuntu in order to work with them wirelessly.

    \item \textbf{Software expertise development.} We developed general
        expertise in C and C++, and specific expertise in the mature libraries
        we chose for websockets (Boost.Beast) and audio recording (PortAudio).
        Notably, we learned to work with the quirks of using PortAudio on the
        Raspberry Pi.  We also developed a good understanding of writing
        programs to perform asynchronous I/O, which is necessary for performant
        audio streaming.

    \item \textbf{Dealing with audio on the Pi.} We also learned more on how to
        deal with audio recording on the Raspberry Pi, which turns out to be a
        non-trivial task due to the lack of documentation and the differences
        between the Linux distro on the Pi and the relatively heavier-weight
        Linux distros we are used to (e.g., Ubuntu).
        The undergraduate RA (Minh) developed a program to print all possible audio
        devices on the Pi, and at the same time, configure ALSA to choose the
        right default audio input device that is needed.

    \item \textbf{audioStreamer.} With the help of our research programmer
        (Vincent), we were able to develop an audio streaming program
        (\texttt{audioStreamer}) that can be started and stopped using control
        messages coming over an MQTT message bus. 

\end{itemize}

Some of the challenges we faced and lessons learned are listed below.

\begin{enumerate}

    \item \textbf{Hardware limitations} The Rasberry Pi took a fair amount of
        work to set up and interface with. This, coupled with its limited
        memory and processing power made it so that it was a non-trivial task
        to program on it (despite it being one of the more accessible
        single-board computers out there). 

        While we were able to get the
        \texttt{audioStreamer} program to compile fine on our laptop and
        desktop computers, the C++ libraries we used for the additional
        capabilities related to communicating with a message bus resulted in
        the compilation time on the Raspberry Pi being excessively long for
        regular development. We are exploring ways to mitigate this, such as
        precompiling non-audio hardware related code on a separate machine, and
        splitting up the code into smaller translation units to better leverage
        incremental compilation.

    \item \textbf{Steep learning curve.} Since we are dealing with a
        resource-limited device like the Pi, we chose to build our programs in
        C++ for efficient resource usage.  However, C++ itself has a steep
        learning curve, as do the libraries we used (Boost.Beast for websockets
        and PortAudio for audio capture).  While the undegraduate research
        assistant (Minh) on the project learned a lot from this task, the fact
        remains that the level of programming expertise required to perform
        this task within the proposed time frame is considerable.

        
\end{enumerate}


\subsection{Current state and future goals}

We are currently pushing forward on two fronts:

\begin{itemize}

    \item We are looking into ways to speed up the compilation of the
        audioStreamer program on the Pi.

    \item In parallel, we are improving the robustness and documentation for
        our ASR agent so that it can be more easily run by researchers outside
        our group.

\end{itemize}

The undegrad RA is now being funded on one of the PI's other grants (ToMCAT) to
continue developing the WASD.
