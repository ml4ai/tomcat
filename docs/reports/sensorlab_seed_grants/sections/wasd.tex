\section{Task 1: Wearable audio streaming device (WASD)}

The goal of this task was to develop a wearable audio streaming device to
study spoken team dialog while bypassing the problem of audio source
separation.

This is a non-trivial task, and was broken down into a few intermediate steps.

\begin{enumerate}
    \item Write a program to 
\end{enumerate}

I was tasked to design a program, called AudioStreamer, that records the audio
and sends those audio chunks to a server, which will then go through a speech
analyzer. The program is also needed to be able to run in a Raspberry Pi, which
has limited memory and processing power, as well as requires a lot of set up to
work.

I have little experience with C and C++ prior to the task. Portaudio and the
Boost.Beast library, which correspond to the recording and the networking parts
of the program, are also difficult to learn for me. Fortunately, with the help
of Vincent Raymond, the program is eventually completed. It runs well on Mac
devices with sufficient memory and processing power.

However, the work is not done, as the Pi is using a Linux distro with very
different setup and has very limited performance. Initially, when I tried to
build the program and run, Portaudio could not find the correct recording
device attached to the Pi. Thus I created a small Portaudio program that prints
out all possible audio devices on the Pi, and, at the same time, tried to
configure ALSA to make it choose the right default audio input device that I
need. Eventually, those changes made the program works on the Pi. However, the
program is stuck after running on the Pi for 1 - 2 seconds (which does not
occur on other Linux or Mac devices). After some changes to the program, I
found out that the problem occurs on the synchronous sending call to
Boost.Beast. Subsequently, the program needs to be edited so that it uses
Portaudio and Boost.Beast asynchronously.

After the first version of the program, Vincent also updated it to include more
features, namely the program can now listen to incoming messages and changes
its mode differently. However, these changes make the program quite large it
takes too long to build, and the Raspberry Pi does not have enough memory for
parallel compilation. Thus it is necessary to splitting up the program or to
compile it first on another machine and transfer it back to the Pi.

Current situation and objective

As of right now, although the program runs well on other machines, it cannot be
built on the Raspberry Pi because of its limited memory and processing power.
Thus the program needs to be split, or built on another machine and then
transferred back to the Pi. I am also currently setting up the ASR agent, which
is the endpoint that the audio is sent to, to test the AudioStreamer program.
After that, it is also necessary to test the working program on the Pi to check
if it runs properly without the problem of the previous version and to change
the Portaudio and Boost.Beast calls to become asynchronous.
