\documentclass[11pt,article,oneside]{memoir}

\usepackage[dvipsnames]{xcolor}
\usepackage[scale=0.85]{cascadia-code}
\usepackage[
    colorlinks,
    linkcolor=Blue,
    citecolor=Blue,
    urlcolor=Blue
]{hyperref}       % hyperlinks
\def\sectionautorefname{\S}
\def\subsectionautorefname{\S}
\usepackage{url}            % simple URL typesetting
\usepackage{booktabs}       % professional-quality tables
\usepackage[tracking]{microtype}      % microtypography
\usepackage[
    natbib,
    style=numeric-comp,
    sorting=none,
    doi=false,
    isbn=false,
    url=true,
    eprint=false,
    maxbibnames=10,
    hyperref
]{biblatex}
\addbibresource{bibliography.bib}
%\renewcommand\bibfont{\small}
%\newbibmacro{string+doiurlisbn}[1]{%
  %\iffieldundef{doi}{%
    %\iffieldundef{url}{%
      %\iffieldundef{isbn}{%
        %\iffieldundef{issn}{%
          %#1%
        %}{%
          %\href{http://books.google.com/books?vid=ISSN\thefield{issn}}{#1}%
        %}%
      %}{%
        %\href{http://books.google.com/books?vid=ISBN\thefield{isbn}}{#1}%
      %}%
    %}{%
      %\href{\thefield{url}}{#1}%
    %}%
  %}{%
    %\href{http://dx.doi.org/\thefield{doi}}{#1}%
  %}%
%}

%\DeclareFieldFormat[article,book,incollection,inproceedings,data]{title}%
    %{\usebibmacro{string+doiurlisbn}{#1}}

%\DeclareFieldFormat*{url}{}
%\DeclareFieldFormat[online]{url}{\mkbibacro{URL}\addcolon\space\url{#1}}

%\DeclareFieldFormat[misc]{title}{\usebibmacro{string+doiurlisbn}{\mkbibemph{#1}}}

\usepackage{fontspec}
% ==========================================================================
% Fonts
% ==========================================================================

\setmainfont[%
  Ligatures = {Common, TeX},
  %ItalicFeatures={Style=Swash},
  SmallCapsFeatures={Letters=SmallCaps,LetterSpace=6},
  Numbers = {OldStyle, Proportional},
]{Arno Pro}
\setsansfont[%
    Numbers = {OldStyle, Proportional},
]{Myriad Pro}
\setmonofont{Menlo}[Scale=MatchLowercase]


% =============================================================================
% Memoir package - layout and styling
% =============================================================================
% Here we set typeblock widths for the main body and the footnotes
\setlxvchars[\normalfont] % about 66 characters per column
\setxlvchars[\normalfont\footnotesize] % about 45 characters per column

% Arno Pro 11pt
% lxvchars: 3.9049805556 inches
% xlvchars: 2.32066152778 inches
\setlrmarginsandblock{1in}{1in}{}
%\setlrmarginsandblock{1in}{3.6in}{}
\setulmarginsandblock{1in}{1in}{*}

% Arno Pro 12pt
% lxvchars: 4.2349340277804 inches
% xlvchars: 2.54297319445 inches
%\setlrmarginsandblock{1in}{3.3in}{}
%\setulmarginsandblock{1in}{1in}{*}


\headstyles{bringhurst}

\sidebarmargin{right}
\setsidebars{*}{2in}{*}{*}{*}{*}
\renewcommand*{\sidebarform}{\raggedright}

% Remove preceding chapter number for section headings.
\counterwithout{section}{chapter}

% Set properties of margin notes, sidecaptioned floats, and footnotes in the
% margin.

% Set sidepar margin
\sideparmargin{outer}

% For Arno Pro 11 pt, change 2.1in to 2.4in
% Set width of footnotes in margin
\setmarginnotes{0.2in}{2.4in}{2\onelineskip}

% Set width of \sidefootnote
\setsidefeet{0.2in}{2.4in}{*}{*}{\tiny}{*}

% Set width of sidecaptions
\setsidecaps{0.2in}{2.4in}

\footnotesinmargin
\sidecapmargin{outer}
\renewcommand*{\sidecapstyle}{\normalfont\footnotesize}
\setsidecappos{t}

% Set footnote style
\renewcommand{\foottextfont}{\footnotesize}


% Make marginfigures centered by default
\setfloatadjustment{marginfigure}{\centering}

% Number subsections
\setsecnumdepth{subsection}

% Add subsections to the table of contents
\settocdepth{subsection}

% Set caption style
\captionstyle[\centering]{\footnotesize}
\captionnamefont{\color{Maroon}\footnotesize}

% Shading
\definecolor{shadecolor}{gray}{0.95}

\tightlists

\checkandfixthelayout

\begin{document}
\chapter*{ToMCAT Summary for ASIST PCR}

\section{Main successes}

\paragraph{Real-time multi-party spoken dialog analysis system}

We developed a system for real-time analysis (real-time transcription, event
extraction, and labeling of sentiment/emotion) of multi-party spoken dialog in
remote experiments. This system is included in the publicly-released ASIST
Study 3 testbed. The system is designed to be
extended to perform other speech analysis tasks, such as entrainment detection
and dialog act classification.

For the ASIST Study 4 testbed, we added a rule-based spellchecking system to
meet the unique requirements of analyzing natural language in Study 4
(reproducibility, real-time output, high precision, and the ability to deal
with domain-specific terms and new types of errors arising from
the informal nature of text chat) that were not met by existing systems.

\paragraph{Rich, multimodal dataset for human-machine teaming} We used the
ASIST Study 3 testbed to collect data from 40 teams of 3 humans each, while
instrumenting participants with a large array of physiological sensors.
Modalities include two kinds of brain scan data— functional near-infrared
spectroscopy (fNIRS) and electroencephalography (EEG), as well as skin
conductance, heart rate, eye tracking, face images, spoken dialog audio data
with automatic speech recognition (ASR) transcriptions, screenshots, gameplay
data, demographic data, and self-report questionnaires. In addition to the
Study 3 USAR dataset also contains data from a set of novel behavioral baseline
tasks.

The dataset is publicly available~\citep{pyarelal2023the}. Notably, this is
the \emph{largest} publicly available fNIRS dataset to date, approximately
\textbf{13.5 times} the size of the next-largest one.

\paragraph{Transition} Our work on this project has led to a cooperative
agreement with ARO. We will be collaborating with ARL scientists to (i) develop
methods to automatically detect closed-loop communication (CLC) within teams,
and (ii) study the effects of CLC on team coordination, planning, and
creativity. We are planning on collecting data using the ASIST Study 3 testbed.

%\section{Lessons learned}

%Besides the inevitable lesson on how everything takes a lot longer than you
%planned, we learned j

\section{Products}

\paragraph{Publications and Datasets} We have published papers at competitive
venues on our public physio dataset~\citep{pyarelal2023the} and dialog act
classification approaches~\citep{qamar-etal-2023-speaking,
miah-etal-2023-hierarchical}.  Additionally, we have published book chapters on
our procedural generation library for voxel maps~\citep{Pyarelal.ea:2022} and
using features at multiple spatial and temporal resolutions to predict human
behavior in real-time~\citep{Zhang.ea:2022c}. We have also disseminated some of
our work through workshops and non-CS/ML/AI conferences---namely, our work on
our ASIST Study 3 event extraction system~\citep{nitschke-etal-2022-rule},
probabilistic modeling of human teams to infer false
beliefs~\citep{Soares.ea:2021}, studying the effect of automated transcription
on emotion, sarcasm, and personality detection from speech
data~\citep{culnan-etal-2021-ire}, and our review on team plan
recognition~\citep{Rieffer_Champlin_2023}.  We have published preprints that
explore reinforcement learning with vector quantized
encoding~\citep{Zhang.ea:2022a} and multi-timescale modeling of human
behavior~\citep{Basavaraj.ea:2022}. Finally, we also have another multimodal
dialog dataset~\citep{multicat} based on annotating ASIST Study 3 data in
preparation that will be submitted for publication in early 2024.

\paragraph{Registrations} We published public preregistrations and results
registrations that detailed our approaches and results ASIST Studes
1--3~\citep{study_1_preregistration_and_results, study_2_preregistration,
    study_2_results, study_3_preregistration, study_3_results}.

\paragraph{Software} Besides the main ToMCAT repository~\citep{tomcat_repo}
that contains the code for our single-player `zombie attack` missions,
procedural generation library and physio data collection, analysis, and
dissemination, we also have public repositories that contain the code for our
ASI~\citep{tomcat_asi}, plan recognition approaches~\citep{tomcat_planrec},
event extraction~\citep{tomcat-text}, and speech
analysis~\citep{tomcat_speech}. Additionally, our updated event extractor and
spellchecker for Study 4 will become public when the Study 4 testbed becomes
public.


\newpage
\printbibliography
\end{document}
