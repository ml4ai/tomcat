\section{Raw data}

In what follows, strings enclosed by a pair of angle brackets (\verb|<>|) indicate
placeholders for more specific strings (i.e., variables).
We break the directory structure for raw ToMCAT data into three parts
\verb|<root>/<study>/<raw_data>|.
As an experiment is run, data is written to the LangLab Linux computer called
``cat''. \verb|<root>| on cat is
\verb|/data/cat|. The data gets mirrored onto the LangLab Linux computer called
``tom'', where \verb|<root>| is \verb|/data/tom|.
This is done by the script \verb|sync_tom_and_cat|, which is called by the script
\verb|pull_tomcat_data|. Ideally, \verb|sync_tom_and_cat| should also be
called from the main driver script as soon as the experiment is over, but
currently we do not do this.

The script \verb|pull_tomcat_data| transfers the data to the IVILAB machine
\textit{i03.cs.arizona.edu}, and makes two backups of it.  Ideally, we would
also create off-site backups, but we do not do this yet. The data is is
written to
\verb|/tomcat_raw_<N>| where \verb|<N>| is 1, 2, 3, or 4, and backed up to
\verb|/tomcat_raw_<N>_B1| and \verb|/tomcat_raw_<N>_B2|.
The script \verb|pull_tomcat_data| then makes links to those multiple data
locations from \verb|/tomcat/data/raw| and provides access via NFS to the compute
servers \textit{laplace.cs.arizona.edu} and \textit{gauss.cs.arizona.edu}.
Thus, on those IVILAB machines, \verb|<ROOT>| is \verb|/tomcat/data/raw|.
The directory structure pattern for \verb|<study>| under the root directory is

\begin{verbatim}
<facility>/experiments/<study>
\end{verbatim}

For this experiment, \verb|<facility>| is \textit{LangLab}, and \verb|<study>|
is \verb|study_3_pilot|.  This study name is a bit misleading, but makes senses
as this study gradually morphed from an initial pilot study to a real one as we
developed the system, but most data is informative.

\verb|<raw_data>| has two subdirectories, ``presession'' and ``group'', containing data
from the presession experiments and main experiments separately. In both cases, we put the
data from one experimental instance into a directory named
\verb|exp_<yyyy>_<mm>_<dd>_<hh>|. Since we only run one main session at a time, and
they last from most of an hour to over three hours, hourly time resolution
suffices to disambiguate them. However, presessions take only 15 to 30 minutes,
and so a presession directory can hold data for multiple participants.

The group session runs are post processed so that all presession data for the
participants in the group are linked from the group data directory. This
matching cannot be done before the group experiment is finished because we do
not know in advance whether there will be no-shows or other last minute
changes.

To further clarify directory naming, on the IVILAB compute servers,
the data for the first valid group session is in:
%
\begin{verbatim}
/tomcat/data/raw/LangLab/experiments/study_3_pilot/group/exp_2022_09_30_10
\end{verbatim}
However, this might be reported  differently because of the linking described
above. Specifically, the previous example is equivalent to:
%
\begin{verbatim}
/tomcat_raw_1/data/LangLab/experiments/study_3_pilot/group/exp_2022_09_30_10
\end{verbatim}
%
In the original data there are some group experiment directories with time
strings earlier than the above example, but those are all preliminary pilot
experiments. We keep the raw data regardless, but all directories with serious
issues are filtered out when we create derived data sets for general
consumption.

\subsection{Raw data structure for group sessions}

As mentioned above, a post-processing step links all needed presession files into
the group experimental runs. We describe the final group session data with
needed presession data included.

%\kobus{Chinmai used <> for variables, except also YYYY, MM, etc. I am not sure
%whether we should try to make it all consistent.}
% Adarsh (2023-07-18): I changed all the variable instances to use <>.

Each of the three participants are associated by the name of the iMac device
they use during the experiment.
The iMac devices are named as lion, tiger, and
leopard. We define
<cat> $\in \{leopard, lion, tiger\}$
and <Cat> $\in \{Leopard, Lion, Tiger\}$,
and use <cat> or <Cat> to represent the three instances.

%\kobus{Some of this is from a very old document. And the rest is from me quickly looking
%at both early and recent data directories. Please check and update!! And include
%details!!!}

\kobus{Also, I understand we changed formats mid stream. We need to specify
these differences here.}

Prior to April 2023, we recorded physio data for each station to a separate
XDF file, and the rest of the data (baseline task observations, Minecraft data,
etc.) to separate files. However, we realized that timestamps were not
synchronized across different XDF files. Furthermore, recording the non-physio
data to separate files made it difficult to synchronize timestamps between
physio and non-physio data. To overcome these limitations, starting in April
2023, we made a substantial change to the recording setup such that all data
was now streamed through LSL and only a single XDF file would be written to at
a time. The only data we excluded from the XDF files was the raw face and screen
capture images---however, we did push the timestamps (corresponding to when the
images were captured) onto LSL, resulting in them being written to the XDF file
as well.

We call the pre-April 2023 setup \emph{v1}, and the subsequent setup \emph{v2}.
For redundancy, we retain the existing mechanisms from v1 that were
recording to non-XDF files---thus, there is some overlap in the directory
structure for the v1 and v2 recording outputs. However, with the exception of
images, the XDF files supersede the non-XDF files for the v2 recording setup.
In the directory structure shown below, we denote which directories are only
present in v1 or v2 data using comments starting with an octothorpe (\verb|#|).

Underneath each experiment (\verb|<raw_data>|) directory (i.e.,\\
\verb|<root>/<study>/group/exp_<yyyy>_<mm>_<dd>_<hh>|), we have the following
file/directory structure:
%
\begin{verbatim}
redcap_data/
    team_data.csv

baseline_tasks/
    affective/
        individual_<cat>_<participantID>_<timestamp>.csv
        individual_<cat>_<participantID>_<timestamp>_metadata.json
        team_<timestamp>.csv
        team_<timestamp>_metadata.json
    finger_tapping/
        <timestamp>.csv
        <timestamp>_metadata.json
    ping_pong/
         competitive_0_<timestamp>.csv
         competitive_0_<timestamp>_metadata.json
         competitive_1_<timestamp>.csv
         competitive_1_<timestamp>_metadata.json
         cooperative_0_<timestamp>.csv
         cooperative_0_<timestamp>_metadata.json
    rest_state/
        <timestamp>.csv

lsl/ # Only for experiments starting April 2023
    block_1.xdf
    block_2.xdf

minecraft/
    MinecraftData_Trial-T00073_ID-<fancy_string>.metadata
    MinecraftData_Trial-Training_ID-<fancy_string>.metadata

<cat>/
    eeg_fnirs_pupil/ # Only for experiments before April 2023
        <cat>_eeg_fnirs_pupil.xdf
    audio/ # Only for sessions on or after 2022-10-07
        ... 3 to 4 .wav files # Prior to 2023-04-17
        block_2/ # On or after 2023-04-17
            ... 3 to 4 .wav files
    face_images/
        ffmpeg.log
        ... a large number of .png files
    presession/
        participant_<ID>.wav
        participant_<ID>Task2.wav
    pupil_recorder/
        000/
            blinks.pldata
            blinks_timestamps.npy
            eye0.intrinsics
            eye0.mp4
            eye0_timestamps.npy
            eye1.intrinsics
            eye1.mp4
            eye1_timestamps.npy
            fixations.pldata
            fixations_timestamps.npy
            gaze.pldata
            gaze_timestamps.npy
            info.player.json
            notify.pldata
            notify_timestamps.npy
            pupil.pldata
            pupil_timestamps.npy
            user_info.csv
            world.intrinsics
            world.mp4
            world_timestamps.npy
    redcap_data/
        <cat>_post_game_survey_data.csv
        <cat>_self_report_data.csv
    screen_shots/
        ffmpeg.log
        ... a large number of .png files

testbed_logs/ # On or after 2022-10-27
    asist_logs_<yyyy>_<mm>_<dd>_<hh>_<mm>_<ss>/

tmp/

data_inventory.log # For sessions starting 2023-04-17
data_inventory.run # For sessions starting 2023-04-17
time_difference.txt # For sessions starting 2023-04-17

trial_info.json
\end{verbatim}

\subsubsection{Description of the files.}

\noindent
Excluding log files, debugging, and other infrastructure files, the format and
the data for each file listed above is detailed as follows: \\


\begin{description}
\item\textbf{redcap\_data/ ...}

    \begin{addmargin}[0em]{0em} % Start of "team_data.csv"
        \textbf{team\_data.csv}\\(comma delimited, 1st row is a header, complex strings double-quoted)\\
        This CSV file is the Team Data record for the experiment exported from the REDCap database.
        The Team Data is info and notes created by the experimenters regarding the experiment.
        The data is inputted into REDCap after the experiment has been completed.
        A summary of data contained in this file is: Team ID, Session Date/Time, Participant's IDs, Absent Participants,
        Experimenters that subbed-in, Problems/Issues with Participants, Problems/Issues with Equipment, and Additional Notes regarding the Session.\\\\
        Team Data Fields:
        \begin{itemize}
            \item \verb|record_id -|\\REDCap Team Data Record ID.
            \item \verb|redcap_survey_identifier - (can be blank)|\\Survey ID that identifies the REDCap Survey Form used to input the Team Data.
            \item \verb|team_data_timestamp - (can be blank)|\\Timestamp of when the Team Data Record was created.
            \item \verb|team_id - [##]|\\Team ID assigned to the Experiment.
            \item \verb|testing_session_date - [yyyy-mm-dd hh:nn] (hh in 24 hour)|\\Experiment Session Date and Time.
            \item \verb|subject_id - [#####, #####, #####]|\\IDs of the Participants that participated in the Experiment. Lion's ID, Tiger's ID, Leopard's ID. (If an experimenter sat-in, the ID will be entered as 99999 for that position).
            \item \verb|real_participant_attend - [No/Yes] (can be blank)|\\Did any of the actual participants with assigned subject IDs not attend?
            \item \verb|real_participant_absent - (can be blank)|\\If \verb|real_participant_attend|=Yes, a list of the subject ID(s) that was scheduled to attend but did not attend.
            \item \verb|research_team_participation - [No/Yes] (can be blank)|\\Did a research team member play as a mock participant during the testing session?
            \item \verb|participants_issues - [No/Yes] (can be blank)|\\Were there any problems/issues with the participants during the testing session?
            \item \verb|participants_issues_details - (can be blank)|\\If \verb|participants_issues|=Yes, bulleted list of participant-related issues during the testing session.
            \item \verb|equipment_issues - [No/Yes] (can be blank)|\\Were there any problems/issues with the equipment during the testing session?
            \item \verb|equipment_issues_details - (can be blank)|\\If \verb|equipment_issues|=Yes, bulleted list of equipment-related issues related during the testing session.
            \item \verb|additional_notes - (can be blank)|\\Any additional notes regarding the testing session.
            \item \verb|team_data_complete - [Incomplete/Unverified/Complete]|\\Status of this Team Data Record.
        \end{itemize}
    \end{addmargin} % End of "team_data.csv"



\textbf{\\\\\\}
\item\textbf{baseline\_tasks/ ...}
    \begin{addmargin}[0em]{0em} % Start of "baseline_tasks/affective/"
        \textbf{affective/ ...}

        \begin{addmargin}[1em]{0em} % Start of "individual_<participantID>_<timestamp>.csv"
            \textbf{individual\_<participantID>\_<timestamp>.csv}\\(semicolon delimited, 1st row is a header)\\
            This CSV file is the Baseline Individual Affective Task Data/Statistics for
            each Participant. The Participant ID is in the of the file name. There will be
            three of these files in the directory. One for each Participant, Lion, Tiger,
            and Leopard. A summary of data contained in this file is: Record Timestamp (in
            Global, Monotonic, and Human formats), Name of Image being shown to the
            Participant, Subject ID (Participant ID), The Participant's Arousal Score, The
            Participant's Valence Score, and the Event Type (\verb|start_affective_task|,
            \verb|show_blank_screen|, \verb|show_cross_screen|, \verb|show_image|,
            \verb|show_rating_screen|, \verb|intermediate_selection|,
            \verb|final_submission|).\\\\
            Baseline Individual Affective Task Fields:
            \begin{itemize}
                \item \verb|time - [##########.######] (in seconds)|\\Unix Time \href{https://www.unixtimestamp.com/}{https://www.unixtimestamp.com/}.
                \item \verb|monotonic_time - [#######.#########] (in seconds)|\\How long since the computer that hosts the task was booted up.
                \item \verb|human_readable_time - [yyyy-mm-ddThh:nn:ss.######Z] (hh in 24 hour)|\\ UTC-0 time in human-readable format.
                \item \verb|image_path -|\\Name of image being shown to the Participant. You can see these images in the code of baseline task.
                \item \verb|subject_id - [#####]|\\Participant ID. (If an experimenter sat-in, the ID will be entered as 99999 for that Participant)
                \item \verb|arousal_score - [-2 to +2]|\\Arousal measure of emotion (calm vs. intense).
                \item \verb|valence_score - [-2 to +2]|\\Valence measure of emotion (unpleasant vs. pleasant).
                \item \verb|event_type -|\\What event and when.
                    (\verb|start_affective_task|, \verb|show_blank_screen|,
                    \verb|show_cross_screen|, \verb|show_image|, \verb|show_rating_screen|,
                    \verb|intermediate_selection|, \verb|final_submission|).
            \end{itemize}
        \end{addmargin} % End of "individual_<participantID>_<timestamp>.csv"


        \textbf{\\\\}
        \begin{addmargin}[1em]{0em} % Start of "individual_<participantID>_<timestamp>_metadata.json"
            \textbf{individual\_<participantID>\_<timestamp>\_metadata.json}\\(JSON data format)\\
            Baseline Individual Affective Task Participant configuration information.
            This is the sequence that the computer shows for each image: blank screen, cross screen, blank screen, image, rating screen.
            The timing for each screen is specified in this JSON file as shown below.\\\\
            Participant Configuration Information JSON File:
            \begin{verbatim}
                {
                    "participant_ids": ["#####"] ("99999" for subbing-in experimenter),
                    "blank_screen_milliseconds": [####],
                    "cross_screen_milliseconds": [####],
                    "individual_image_timer": [##.#] (in seconds),
                    "individual_rating_timer": [##.#] (in seconds),
                    "team_image_timer": [##.#] (in seconds),
                    "team_discussion_timer": [##.#] (in seconds),
                    "team_rating_timer": [##.#] (in seconds)
                }
            \end{verbatim}
        \end{addmargin} % End of "individual_<participantID>_<timestamp>_metadata.json"


        \textbf{\\\\}
        \begin{addmargin}[1em]{0em} % Start of "team_<timestamp>.csv"
            \textbf{team\_<timestamp>.csv}\\(semicolon delimited, 1st row is a header)\\
            This CSV file is the Baseline Team Affective Task Data/Statistics. A summary of data contained in this file is: Record Timestamps (in Global, Monotonic, and Human formats), Name of Image being shown to the Participants, Subject ID (Participant ID), The Participant's Arousal Score (blank if this participant was not selected to score this image), The Participant's Valence Score (blank if this participant was not selected to score this image), and the Event Type
            (\verb|start_affective_task|, \verb|show_blank_screen|,
            \verb|show_cross_screen|, \verb|show_image|, \verb|show_rating_screen|,
            \verb|intermediate_selection|, \verb|final_submission|).\\\\
            Baseline Team Affective Task Fields:
            \begin{itemize}
                \item \verb|time| - [\#\#\#\#\#\#\#\#\#\#.\#\#\#\#\#\#] (in seconds)\\Unix Time \href{https://www.unixtimestamp.com/}{https://www.unixtimestamp.com/}.
                \item \verb|monotonic_time| - [\#\#\#\#\#\#\#.\#\#\#\#\#\#\#\#\#] (in seconds)\\How long since the computer that hosts the task was booted up.
                \item \verb|human_readable_time| - [yyyy-mm-ddThh:nn:ss.\#\#\#\#\#\#Z] (hh in 24 hour)\\ UTC-0 time in human-readable format.
                \item \verb|image_path| - [Team\#\#\#.jpg]\\Name of image being shown to the participants. You can see these images in the code of baseline task.
                \item \verb|subject_id| - [\#\#\#\#\#]\\Participant ID. (If an experimenter sat-in, the ID will be entered as 99999 for that Participant)
                \item \verb|arousal_score| - [-2 to +2]\\Arousal measure of emotion (calm vs. intense, will be blank if this participant was not selected to score this image).
                \item \verb|valence_score| - [-2 to +2]\\Valence measure of emotion (unpleasant vs. pleasant, will be blank if this participant was not selected to score this image).
                \item \verb|event_type| -\\What event and when.
                    (\verb|start_affective_task|, \verb|show_blank_screen|,
                    \verb|show_cross_screen|, \verb|show_image|, \verb|show_rating_screen|,
                    \verb|intermediate_selection|, \verb|final_submission|).
            \end{itemize}
        \end{addmargin} % End of "team_<timestamp>.csv"


        \textbf{\\\\}
        \begin{addmargin}[1em]{0em} % Start of "team\_<timestamp>\_metadata.json"
            \textbf{team\_<timestamp>\_metadata.json}\\(JSON data format)\\
            Baseline Team Affective Task Participant configuration information.
            This is the sequence that the computer shows for each image: blank screen, cross screen, blank screen, image, rating screen.
            The timing for each screen is specified in this JSON file as shown below.\\\\
            Team Configuration Information JSON File:
            \begin{verbatim}
                {
                    "participants_ids": [
                        ("#####","#####","#####"; "99999" for experimenter)
                        "<lion_participant_id>",
                        "<tiger_participant_id>",
                        "<leopard_participant_id>"
                    ],
                    "blank_screen_milliseconds": [####],
                    "cross_screen_milliseconds": [####],
                    "individual_image_timer": [##.#] (in seconds),
                    "individual_rating_timer": [##.#] (in seconds),
                    "team_image_timer": [##.#] (in seconds),
                    "team_discussion_timer": [##.#] (in seconds),
                    "team_rating_timer": [##.#] (in seconds)
                }
            \end{verbatim}
        \end{addmargin} % End of "team\_<timestamp>\_metadata.json"

    \end{addmargin} % End of "baseline_tasks/affective/" 


    \textbf{\\\\}
    \begin{addmargin}[0em]{0em} % Start of "baseline_tasks/finger_tapping/"
        \textbf{finger\_tapping/ ...}

        \begin{addmargin}[1em]{0em} % Start of "<timestamp>.csv"
            \textbf{<timestamp>.csv}\\(semicolon delimited, 1st row is a header)\\
            This CSV file is the Baseline Finger Tapping Task Data/Statistics.
            A summary of data contained in this file is: Record Timestamp (Unix Time, Monotonic, and Human-readable formats),
            Row Data Event (team, individual), Countdown Timer (intiger - 10 to 0), Was a Tap on Keyboard recorded for each participant (0 = no-tap, 1 = tap).
            The last three column (Fields) names for the Tap Data are the IDs of the Participants (\verb|<lion_participant_id>|, \verb|<tiger_participant_id>|,
            \verb|<leopard_participant_id>|, If an experimenter sat-in, the column name will be "99999" for that Participant).\\\\
            Baseline Individual Affective Task Fields:
            \begin{itemize}
                \item \verb|time| - [\#\#\#\#\#\#\#\#\#\#.\#\#\#\#\#\#] (in seconds)\\Unix Time \href{https://www.unixtimestamp.com/}{https://www.unixtimestamp.com/}.
                \item \verb|monotonic_time| - [\#\#\#\#\#\#\#.\#\#\#\#\#\#\#\#\#] (in seconds)\\How long since the computer that hosts the task was booted up.
                \item \verb|human_readable_time| - [yyyy-mm-ddThh:nn:ss.\#\#\#\#\#\#Z] (hh in 24 hour)\\ UTC-0 time in human-readable format.
                \item \verb|event_type| -\\What event and when. (team, individual).
                \item \verb|countdown_timer| - [\#\#] (intiger - 10 to 0)\\Countdown Timer.
                \item \verb|<lion_participant_id>| - [0 or 1]\\Tap on keyboard from Lion (0 = no-tap, 1 = tap).
                \item \verb|<tiger_participant_id>| - [0 or 1]\\Tap on keyboard from Tiger (0 = no-tap, 1 = tap).
                \item \verb|<leopard_participant_id>| - [0 or 1]\\Tap on keyboard from Leopard (0 = no-tap, 1 = tap).
            \end{itemize}
        \end{addmargin} % End of "<timestamp>.csv"


        \textbf{\\\\}
        \begin{addmargin}[1em]{0em} % Start of "<timestamp>_metadata.json"
            \textbf{<timestamp>\_metadata.json}\\(JSON data format)\\
            Baseline Finger Tapping Task configuration information. The configuration
            information in this file: \verb|participants_ids| \verb|session|
            \verb|seconds_per_session| \verb|seconds_count_down| \verb|square_width| and
            \verb|count_down_message|.\\\\
            Finger Tapping Configuration Information JSON File:
            \begin{verbatim}
                {
                    "participants_ids": [
                        ("#####","#####","#####"; "99999" for experimenter)
                        "<lion_participant_id>",
                        "<tiger_participant_id>",
                        "<leopard_participant_id>"
                    ],
                    "session": [ (typical: "0,1,0,1")
                        (0 or 1),
                        (0 or 1),
                        (0 or 1),
                        (0 or 1)
                    ],
                    "seconds_per_session": [ (typical: "10.0" for all)
                        ##.#,
                        ##.#,
                        ##.#,
                        ##.#
                    ],
                    "seconds_count_down": [##.#] (typical: "10.0"),
                    "square_width": [###] (typical: "200")
                    "count_down_message": ["string"]
                    (example: "Practice session: Press SPACEBAR and observe the squares")
                }
            \end{verbatim}
        \end{addmargin} % End of "<timestamp>_metadata.json"

    \end{addmargin} % End of "baseline_tasks/finger_tapping/" 


    \textbf{\\\\}
    \begin{addmargin}[0em]{0em} % Start of "baseline_tasks/ping_pong/"
        \textbf{ping\_pong/ ...}

        \begin{addmargin}[1em]{0em} % Start of "competitive_<team>_<timestamp>.csv"
            \textbf{competitive\_<team>\_<timestamp>.csv}\\(semicolon delimited, 1st row is a header)\\
            This CSV file is for the Baseline Competitive Ping-Pong Task Data/Statistics.
            The <team> in the file name is "0" for Lion vs Tiger and "1" for Leopard vs Cheetah.
            (If an experimenter sat-in, the column name will be "99999" for that Participant).
            (For <team> = "1" in the file name, the participant2 ID will always be "exp").
            A summary of data contained in this file is: Record Timestamp (Unix Time, Monotonic, and Human-readable formats),
            Score on Left, Score on Right (For <team> = "1" in the file name, right score will be for experimenter on "Cheetah"), Game Started (False = countdown for game to start, True = game has started),
            Ball's X Coordinates, Ball's Y Coordinates, Participant 1 Paddle X Coordinates, Participant 1 Paddle Y Coordinates,
            Participant 2 Paddle X Coordinates , Participant 2 Paddle Y Coordinates, Seconds Timer on Screen (If game has not started, \verb|started = False|,
            the \verb|seconds| will count down from 10 to 0. If game has started, \verb|started = True|, the \verb|seconds| will count down from 120 to 0.)\\\\
            Baseline Competitive Ping-Pong Task Fields:
            \begin{itemize}
                \item \verb|time - [##########.######] (in seconds)|\\Unix Time \href{https://www.unixtimestamp.com/}{https://www.unixtimestamp.com/}.
                \item \verb|monotonic_time - [#######.#########] (in seconds)|\\How long since the computer that hosts the task was booted up.
                \item \verb|human_readable_time - [yyyy-mm-ddThh:nn:ss.######Z] (hh in 24h)|\\ UTC-0 time in human-readable format.
                \item \verb|score_left - [##]|\\ Current score for left team participant.
                \item \verb|score_right - [##]|\\ Current score for right team participant (For <team> = "1" in the file name, right score will be for experimenter on "Cheetah").
                \item \verb|started - [False or True]|\\Has the Ping-Pong game started.
                \item \verb|ball_x - [####]| - \\The ball's X coordinate on the screen.
                \item \verb|ball_y - [####]| - \\The ball's Y coordinate on the screen.
                \item \verb|<participant1_id>_x - [####]| -\\Participant1's (left team) paddle's X coordinate on the screen.
                \item \verb|<participant1_id>_y - [####]| -\\Participant1's (left team) paddle's Y coordinate on the screen.
                \item \verb|<participant2_id>_x - [####]| -\\Participant2's (right team) paddle's X coordinate on the screen (For <team> = "1" in the file name, the participant2 ID will always be "exp").
                \item \verb|<participant2_id>_y - [####]| -\\Participant2's (right team) paddle's Y coordinate on the screen (For <team> = "1" in the file name, the participant2 ID will always be "exp").
                \item \verb|seconds - [###]|\\Seconds left in game (120 counts down to 0).
            \end{itemize}
        \end{addmargin} % End of "competitive_<team>_<timestamp>.csv"


        \textbf{\\\\}
        \begin{addmargin}[1em]{0em} % Start of "competitive_<team>_<timestamp>_metadata.json"
            \textbf{competitive\_<team>\_<timestamp>\_metadata.json}\\(JSON data format)\\
            Baseline Competitive Ping-Pong Configuration Information.
            (For <team> = "1" in the file name, the participant2 ID will always be "exp").
            The configuration information in this file: \verb|left_team participant ID|, \verb|right_team participant ID|,
            \verb|client_window_height|, \verb|client_window_width|, \verb|session_time_seconds|,
            \verb|seconds_count_down|, \verb|count_down_message|, \verb|paddle_width|, \verb|paddle_height|,
            \verb|ai_paddle_max_speed|, \verb|paddle_speed_scaling|, \verb|paddle_max_speed|, \verb|ball_x_speed|,
            \verb|ball_bounce_on_paddle_scale|.\\\\
            Baseline Competitive Ping-Pong Configuration Information JSON File:
            \begin{verbatim}
                {
                    "left_team": [
                        "#####" (left team participant ID, "99999" for experimenter)
                    ],
                    "right_team": [
                        "#####" (right team participant ID, "99999" for experimenter, "exp" for <team> = "1" in file name.)
                    ],
                    "client_window_height": [####] (typical: 1440),
                    "client_window_width": [####] (typical: 2560),
                    "session_time_seconds": [###.#] (typical: 120.0),
                    "seconds_count_down": [##.#] (typical: 10.0),
                    "count_down_message": ["string"] (typical:"Move the mouse to move the blue paddle"),
                    "paddle_width": [##] (typical: 20),
                    "paddle_height": [###] (typical: 120),
                    "ai_paddle_max_speed": [##] (typical: 13),
                    "paddle_speed_scaling": [#.#] (typical: 0.6),
                    "paddle_max_speed": [###.#] (typical: null),
                    "ball_x_speed": [##] (typical: 9),
                    "ball_bounce_on_paddle_scale": [#.##] (typical: 0.25)
                }
            \end{verbatim}
        \end{addmargin} % End of "competitive_<team>_<timestamp>_metadata.json"


        \textbf{\\\\}
        \begin{addmargin}[1em]{0em} % Start of "cooperative_0_<timestamp>.csv"
            \textbf{cooperative\_0\_<timestamp>.csv}\\(semicolon delimited, 1st row is a header)\\
            This CSV file is for the Baseline Cooperative Ping-Pong Task Data/Statistics.
            For the Cooperative Ping-Pong Task, the participants on Lion, Tiger and Leopard play together as a team and play against the AI machine.
            (If an experimenter sat-in, the column name will be "99999" for that Participant).
            A summary of data contained in this file is: Record Timestamp (Unix Time, Monotonic, and Human-readable formats),
            Score on Left, Score on Right, Game Started (False = countdown for game to start, True = game has started),
            Ball's X Coordinates, Ball's Y Coordinates, Participant 1 Paddle X Coordinates, Participant 1 Paddle Y Coordinates,
            Participant 2 Paddle X Coordinates, Participant 2 Paddle Y Coordinates, Seconds Timer on Screen (If game has not started, \verb|started = False|,
            the \verb|seconds| will count down from 10 to 0. If game has started, \verb|started = True|, the \verb|seconds| will count down from 120 to 0.)\\\\
            Baseline Cooperative Ping-Pong Task Fields:
            \begin{itemize}
                \item \verb|time - [##########.######] (in seconds)|\\Unix Time \href{https://www.unixtimestamp.com/}{https://www.unixtimestamp.com/}.
                \item \verb|monotonic_time - [#######.#########] (in seconds)|\\How long since the computer that hosts the task was booted up.
                \item \verb|human_readable_time - [yyyy-mm-ddThh:nn:ss.######Z] (hh in 24h)|\\UTC-0 time in human-readable format.
                \item \verb|score_left - [##]|\\Current score for left team.
                \item \verb|score_right - [##]|\\Current score for "AI" team.
                \item \verb|started - [False or True]|\\Has the Ping-Pong game started.
                \item \verb|ball_x - [####]|\\The ball's X coordinate on the screen.
                \item \verb|ball_y - [####]| - \\The ball's Y coordinate on the screen.
                \item \verb|<left_team_participant1_id>_x - [####]| -\\Participant1's (left team) paddle's X coordinate on the screen.
                \item \verb|<left_team_participant1_id>_y - [####]| -\\Participant1's (left team) paddle's Y coordinate on the screen.
                \item \verb|<left_team_participant2_id>_x - [####]| -\\Participant2's (left team) paddle's X coordinate on the screen.
                \item \verb|<left_team_participant2_id>_y - [####]| -\\Participant2's (left team) paddle's Y coordinate on the screen.
                \item \verb|<left_team_participant3_id>_x - [####]| -\\Participant3's (left team) paddle's X coordinate on the screen.
                \item \verb|<left_team_participant3_id>_y - [####]| -\\Participant3's (left team) paddle's Y coordinate on the screen.
                \item \verb|ai_x - [####]| -\\AI's (right team) paddle's X coordinate on the screen.
                \item \verb|ai_y - [####]| -\\AI's (right team) paddle's Y coordinate on the screen.
                \item \verb|seconds - [###]|\\Seconds left in game (120 counts down to 0).
            \end{itemize}
        \end{addmargin} % End of "cooperative_0_<timestamp>.csv"


        \textbf{\\\\}
        \begin{addmargin}[1em]{0em} % Start of "cooperative_0_<timestamp>_metadata.json"
            \textbf{cooperative\_0\_<timestamp>\_metadata.json}\\(JSON data format)\\
            Baseline Cooperative Ping-Pong Configuration Information. The configuration
            information in this file: \verb|left_team (participant IDs for Lion, Tiger, and Leopard)|, \verb|right_team ("ai")|,
            \verb|client_window_height|, \verb|client_window_width|, \verb|session_time_seconds|,
            \verb|seconds_count_down|, \verb|count_down_message|, \verb|paddle_width|, \verb|paddle_height|,
            \verb|ai_paddle_max_speed|, \verb|paddle_speed_scaling|, \verb|paddle_max_speed|, \verb|ball_x_speed|,
            \verb|ball_bounce_on_paddle_scale|.\\\\
            Baseline Cooperative Ping-Pong Configuration Information JSON File:
            \begin{verbatim}
                {
                    "left_team": [
                        "#####", (left team member 1 ID, "99999" for experimenter)
                        "#####", (left team member 2 ID, "99999" for experimenter)
                        "#####"  (left team member 3 ID, "99999" for experimenter)
                    ],
                    "right_team": [
                        "ai"
                    ],
                    "client_window_height": [####] (typical: 1440),
                    "client_window_width": [####] (typical: 2560),
                    "session_time_seconds": [###.#] (typical: 120.0),
                    "seconds_count_down": [##.#] (typical: 10.0),
                    "count_down_message": ["string"] (typical:"Move the mouse to move the blue paddle"),
                    "paddle_width": [##] (typical: 20),
                    "paddle_height": [###] (typical: 90),
                    "ai_paddle_max_speed": [##] (typical: 20),
                    "paddle_speed_scaling": [#.#] (typical: 0.6),
                    "paddle_max_speed": [###.#] (typical: null),
                    "ball_x_speed": [##] (typical: 12),
                    "ball_bounce_on_paddle_scale": [#.##] (typical: 0.4)
                }
            \end{verbatim}
        \end{addmargin} % End of "cooperative_0_<timestamp>_metadata.json"

    \end{addmargin} % End of "baseline_tasks/ping_pong/" 


    \textbf{\\\\}
    \begin{addmargin}[0em]{0em} % Start of "baseline_tasks/rest_state/"
        \textbf{rest\_state/ ...}

        \begin{addmargin}[1em]{0em} % Start of "<timestamp>.csv"
            \textbf{<timestamp>.csv}\\(semicolon delimited, 1st row is a header)\\
            This CSV file records the Start Time and End Time for the Baseline Rest State.
            A summary of data contained in this file is: Record Timestamp (Unix Time, Monotonic, and Human-readable formats),
            and Event Type (\verb|"start_rest_state" or "end_rest_state"|).\\\\
            Baseline Rest State Timestamp CSV Fields:
            \begin{itemize}
                \item \verb|time - [##########.######] (in seconds)|\\Unix Time \href{https://www.unixtimestamp.com/}{https://www.unixtimestamp.com/}.
                \item \verb|monotonic_time - [#######.#########] (in seconds)|\\How long since the computer that hosts the task was booted up.
                \item \verb|human_readable_time - [yyyy-mm-ddThh:nn:ss.######Z] (hh in 24h)|\\UTC-0 time in human-readable format.
                \item \verb|event_type - [string]|\\Start or End of Rest State (\verb|"start_rest_state" or "end_rest_state"|).
            \end{itemize}
        \end{addmargin} % End of "<timestamp>.csv"

    \end{addmargin} % End of "baseline_tasks/rest_state/"
    
% End of "baseline_tasks/"



\textbf{\\\\\\}
\item\textbf{lsl/ ... \textit{(Only for experiments starting April 2023)}}
\begin{addmargin}[0em]{0em}
    \textbf{block\_1.xdf}\\(Extensible Data Format XDF, binary file format)\\
    The \verb|block_1.xdf| contains data files and data streams for the Baseline Tasks portion of the Experiment.
    You must use a XDF viewer program to view or extract the data contained in this file.
    Some common software packages used to view or extract data from this XDF file are: MNE-Python, Matplotlib, and Qtgraph.
    A summary of data recorded in this XDF file is: fNIRS LSL Streams, EEG LSL Streams, Baseline Data,
    Filenames of the Face and Screen Images, and Pupil Data.\\\\
    Data files and streams contained in the \verb|block_1.xdf|:
    \begin{itemize}
        \item \verb|fNIRS LSL Streams| -\\fNIRS LSL Streams being transmitted from the "NIRx - Aurora" software programs running on the "fNIRS Server Computer" during the Baseline Tasks portion of the Experiment for participants on Lion, Tiger, and Leopard.
        \item \verb|EEG LSL Streams| -\\EEG LSL Streams being transmitted from the "Brain Vision - LSL-actiChamp" software programs running on the "EEG Server Computer" during the Baseline Tasks portion of the Experiment for participants on Lion, Tiger, and Leopard.
        \item \verb|Baseline Data for all Tasks| -\\All records that are outputted to the Baseline Tasks CSV files are also recorded in this XDF file for all Baseline Tasks.
        \item \verb|Filenames of the Face Images| -\\The Filenames of all Face Images created during the Baseline Tasks portion of the Experiment for participants on Lion, Tiger, and Leopard.
        \item \verb|Filenames of the Screen Images| -\\The Filenames of all Face Images created during the Baseline Tasks portion of the Experiment for participants on Lion, Tiger, and Leopard.
        \item \verb|Pupil Data| -\\Pupil Data files recorded from the "Pupil Labs - Pupil Capture" software programs running on the participant's iMacs, Lion, Tiger, and Leopard during the Baseline Tasks portion of the Experiment.
    \end{itemize}


    \textbf{\\\\}
    \textbf{block\_2.xdf}\\(Extensible Data Format XDF, binary file format)\\
    The \verb|block_2.xdf| contains data files and data streams for the Minecraft portion of the Experiment.
    You must use a XDF viewer program to view or extract the data contained in this file.
    Some common software packages used to view or extract data from this XDF file are: MNE-Python, Matplotlib, and Qtgraph.
    A summary of data recorded in this XDF file is: fNIRS LSL Streams, EEG LSL Streams, Individual and Central Audio,
    Filenames of the Face and Screen Images, and Pupil Data.\\\\
    Data files and streams contained in the \verb|block_2.xdf|:
    \begin{itemize}
        \item \verb|fNIRS LSL Streams| -\\fNIRS LSL Streams being transmitted from the "NIRx - Aurora" software programs running on the "fNIRS Server Computer" during the Minecraft portion of the Experiment for participants on Lion, Tiger, and Leopard.
        \item \verb|EEG LSL Streams| -\\EEG LSL Streams being transmitted from the "Brain Vision - LSL-actiChamp" software programs running on the "EEG Server Computer" during the Minecraft portion of the Experiment for participants on Lion, Tiger, and Leopard.
        \item \verb|Minecraft Messages| -\\A series of JSON strings recording the messages sent to and from participants during the three Minecraft missions, Training, Saturn A, and Saturn B.
        \item \verb|Individual Audio| -\\Audio Signals captured during the Minecraft portion of the Experiment from each participant's microphone, Lion, Tiger, and Leopard.
        \item \verb|Central Audio| -\\This is Audio File from the central array microphone located in the center of the experiment room that picks up all audio in the room during the experiment.
        \item \verb|Filenames of the Face Images| -\\The Filenames of all Face Images created during the Minecraft portion of the Experiment for participants on Lion, Tiger, and Leopard.
        \item \verb|Filenames of the Screen Images| -\\The Filenames of all Face Images created during the Minecraft portion of the Experiment for participants on Lion, Tiger, and Leopard.
        \item \verb|Pupil Data| -\\Pupil Data files recorded from the "Pupil Labs - Pupil Capture" software programs running on the participant's iMacs, Lion, Tiger, and Leopard during the Minecraft portion of the Experiment.
    \end{itemize}

\end{addmargin}



\textbf{\\\\\\}
\item\textbf{<cat>/ ...} % Start of Experiment Directory "<cat>/"

\begin{addmargin}[0em]{0em} % Start of Sub Directory "eeg_fnirs_pupil/" 
    \textbf{eeg\_fnirs\_pupil/ ... \textit{(Only for experiments before April 2023)}}

    \begin{addmargin}[1em]{0em} % Start of File "<cat>_eeg_fnirs_pupil.xdf"
        \textbf{<cat>\_eeg\_fnirs\_pupil.xdf}\\
        (Extensible Data Format XDF, binary file format)\\
        The \verb|<cat>_eeg_fnirs_pupil.xdf| contains <cat> data files and data streams for the Experiment.
        You must use a XDF viewer program to view or extract the data contained in this file.
        Some common software packages used to view or extract data from this XDF file are: MNE-Python, Matplotlib, and Qtgraph.
        A summary of data recorded in this XDF file is: fNIRS LSL Streams, EEG LSL Streams, and Pupil Data.\\\\
        Data files and streams contained in the \verb|<cat>_eeg_fnirs_pupil.xdf|:
        \begin{itemize}
            \item \verb|fNIRS LSL Streams| -\\fNIRS LSL Streams being transmitted from the "NIRx - Aurora" software programs running on the "fNIRS Server Computer" during the Experiment for participants on Lion, Tiger, and Leopard.
            \item \verb|EEG LSL Streams| -\\EEG LSL Streams being transmitted from the "Brain Vision - LSL-actiChamp" software programs running on the "EEG Server Computer" during the Experiment for participants on Lion, Tiger, and Leopard.
            \item \verb|Pupil Data| -\\Pupil Data files recorded from the "Pupil Labs - Pupil Capture" software programs running on the participant's iMacs, Lion, Tiger, and Leopard during the Experiment.
        \end{itemize}
    \end{addmargin} % End of File "<cat>_eeg_fnirs_pupil.xdf"

\end{addmargin} % End of Sub Directory "eeg_fnirs_pupil/"


\textbf{\\\\\\}
\begin{addmargin}[0em]{0em} % Start of Sub Directory "face_images/" 
    \textbf{face\_images/ ... \textit{(Only for experiments before April 2023)}}

    \begin{addmargin}[1em]{0em} % Start of File "Face Image Files <yyyy-mm-ddThh_mm_ss.#########Z>.png"
        % 2023-02-21T22_50_04.976190450Z.png
        \textbf{Face Image Files <yyyy-mm-dd>T<hh\_mm\_ss.sssssssss>Z.png}\\
        (PNG - Portable Network Graphic, raster image file)\\
        Participant Face Image files recorded from the built-in "web camera" on the Lion, Tiger, and Leopard iMacs.
        The files are recorded at a frequency of 10Hz and have a resolution of 1920 x 1080 pixels.
        Typically, there will be over 50,000 files per <cat> for the experiment. 
    \end{addmargin} % End of File "Face Image Files <yyyy-mm-ddThh_mm_ss.#########Z>.png"

\end{addmargin} % End of Sub Directory "face_images/"


\textbf{\\\\\\}
\begin{addmargin}[0em]{0em} % Start of Sub Directory "face_images/block_1/" 
    \textbf{face\_images/block\_1/ ... \textit{(Only for experiments starting April 2023)}}

    \begin{addmargin}[1em]{0em} % Start of Files "Face Image Files, block_1, <seq ######>_<yyyy-mm-dd>_<hh_mm_ss.sss.AM/PM>~<ms from last image ###>.png"
        % 017627_2023-04-28_18-44-18.089.PM~113.png
        \textbf{Face Image Files (block\_1):\\<seq \#\#\#\#\#\#>\_<yyyy-mm-dd>\_<hh\_mm\_ss.sss.AM/PM>~<ms last image \#\#\#>.png}\\
        (PNG - Portable Network Graphic, raster image file)\\
        Participant Face Image files (block\_1) recorded from the built-in "web camera" on the Lion, Tiger,
        and Leopard iMacs during the Baseline Tasks portion of the Experiment.
        The files are recorded at a frequency of 10Hz and have a resolution of 1920 x 1080 pixels.
        Typically, there will be over 15,000 files in this directory per <cat> for the experiment. 
    \end{addmargin} % End of Files "Face Image Files, block_1, <seq ######>_<yyyy-mm-dd>_<hh_mm_ss.sss.AM/PM>~<ms from last image ###>.png"

\end{addmargin} % End of Sub Directory "face_images/block_1/"


\textbf{\\\\\\}
\begin{addmargin}[0em]{0em} % Start of Sub Directory "face_images/block_2/" 
    \textbf{face\_images/block\_2/ ... \textit{(Only for experiments starting April 2023)}}

    \begin{addmargin}[1em]{0em} % Start of Files "Face Image Files, block_2, <seq ######>_<yyyy-mm-dd>_<hh_mm_ss.sss.AM/PM>~<ms from last image ###>.png"
        % 017627_2023-04-28_18-44-18.089.PM~113.png
        \textbf{Face Image Files (block\_2):\\<seq \#\#\#\#\#\#>\_<yyyy-mm-dd>\_<hh\_mm\_ss.sss.AM/PM>~<ms last image \#\#\#>.png}\\
        (PNG - Portable Network Graphic, raster image file)\\
        Participant Face Image files (block\_2) recorded from the built-in "web camera" on the Lion, Tiger,
        and Leopard iMacs during the Minecraft portion of the Experiment.
        The files are recorded at a frequency of 10Hz and have a resolution of 1920 x 1080 pixels.
        Typically, there will be over 30,000 files in this directory per <cat> for the experiment. 
    \end{addmargin} % End of Files "Face Image Files, block_2, <seq ######>_<yyyy-mm-dd>_<hh_mm_ss.sss.AM/PM>~<ms from last image ###>.png"

\end{addmargin} % End of Sub Directory "face_images/block_2/"


\textbf{\\\\\\}
\begin{addmargin}[0em]{0em} % Start of Sub Directory "screenshots/" 
    \textbf{screenshots/ ... \textit{(Only for experiments before April 2023)}}

    \begin{addmargin}[1em]{0em} % Start of File "Screenshot Image Files <yyyy-mm-ddThh_mm_ss.#########Z>.png"
        % 2023-02-21T22_50_04.976190450Z.png
        \textbf{Screenshot Image Files <yyyy-mm-dd>T<hh\_mm\_ss.sssssssss>Z.png}\\
        (PNG - Portable Network Graphic, raster image file)\\
        Screenshot Image files recorded from the participant's iMac (Lion, Tiger, or Leopard).
        The files are recorded at a frequency of 10Hz and have a resolution of 2560 x 1440 pixels.
        Typically, there will be over 50,000 files per <cat> for the experiment. 
    \end{addmargin} % End of File "Screenshot Image Files <yyyy-mm-ddThh_mm_ss.#########Z>.png"

\end{addmargin} % End of Sub Directory "screenshots/"

% Start of Experiment Directory "<cat>/"

\end{description}
