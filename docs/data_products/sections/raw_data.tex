\section{Raw data}

In what follows, upper case indicates
a placeholder for a more specific string (i.e., variables), whereas lower case or mixed case
indicates actual strings (i.e., verbatim).
We break the directory structure for raw ToMCAT data into three parts
\code{ROOT\/STUDY\/RAW\_DATA}. 
As an experiment is run, data is written to the LangLab linux computer called
``cat''. ROOT on cat is
\code{/data/cat}. The data gets mirrored onto the LangLab linux computer called
``tom'', where ROOT is \code{/data/tom}. 
This is done by the script \code{sync\_tom\_and\_cat}, which is called by the script 
\code{pull\_tomcat\_data}. Ideally, \code{sync\_tom\_and\_cat} should also be
called from the main driver script as soon as the experiment is over, but
currently we do not do this. 

The script \code{pull\_tomcat\_data} transfers the data to the IVILAB machine
\textit{i03.cs.arizona.edu}, and makes two backups of it.  Ideally, we would
also create off-site backups, but we do not do this yet. The data is is
written to
\code{/tomcat\_raw\_NNN} where NNN is 1, 2, 3, or 4, and backed up to 
\code{/tomcat\_raw\_NNN\_B1} and \code{/tomcat\_raw\_NNN\_B2}.  
The script \code{pull\_tomcat\_data} then makes links to those multiple data
locations from \code{/tomcat/data/raw} and provides access via NFS to the compute
servers \textit{laplace.cs.arizona.edu} and \textit{gauss.cs.arizona.edu}.
Thus, on those IVILAB machines, ROOT is \code{/tomcat/data/raw}.

The directory structure pattern for STUDY under the root directory is
\code{FACILTY/experiments/STUDY\_NAME}. For this experiment, FACILITY is
\textit{LangLab}, and STUDY\_NAME is \textit{study\_3\_pilot}.
This study name is a bit misleading, but makes senses as this study
gradually morphed from an initial pilot study to a real one as we developed the
system, but most data is informative. 

RAW\_DATA has two subdirectories, ``presession'' and ``group'', containing data
from the presession experiments and main experiments separately. In both cases, we put the
data from one experimental instance into a directory named
\code{exp\_YEAR\_MM\_DD\_HOUR}. Since we only run one main session at a time, and
they last from most of an hour to over three hours, hourly time resolution
suffices to disambiguate them. However, presessions take only 15 to 30 minutes,
and so a presession directory can hold data for multiple participants. 

The group session runs are post processed so that all presession data for the
participants in the group are linked from the group data directory. This
matching cannot be done after the group experiment is finished because we do not
know in advance whether there will be no-shows or other last minute changes. 

To further clarify directory naming, on the IVILAB cpu servers,
the data for the first valid group session 
is in:
\begin{lstlisting}
/tomcat/data/raw/LangLab/experiments/study\_3\_pilot/group/exp\_2022\_09\_30\_10
\end{lstlisting}
However, this might be reported  differently because of the linking described
above. Specifically, the previous example is equivalent to :
\begin{lstlisting}
/tomcat_raw_1/data/LangLab/experiments/study\_3\_pilot/group/exp\_2022\_09\_30\_10
\end{lstlisting}
In the original data there are some group experiment directories with time
strings earlier than the above example, but those are all preliminary pilot
experiments. We keep the raw data irregardless, but all directories with serious
issues are filtered out when we create derived data sets for general
consumption.

\subsection{Raw data structure for group sessions.}

As mentioned above, a post-processing step links all needed presession files into
the group experimental runs. We describe the final group session data with
needed presession data included. 

\kobus{Chinmai used <> for variables, except also YYYY, MM, etc. I am not sure
whether we should try to make it all consistent.}

Each of the three participants are associated by the name of the iMac device
they use during the experiment.
The iMac devices are named as lion, tiger, and
leopard. We define 
<cat> $\in \{leopard, lion, tiger\}$ 
and <Cat> $\in \{Leopard, Lion, Tiger\}$,
and use <cat> or <Cat> to represent the three instances. 

\kobus{Some of this is from a very old document. And the rest is from me quickly looking
at both early and recent data directories. Please check and update!! And include
details!!!}

\kobus{Also, I understand we changed formats mid stream. We need to specify
these differences here.}

Underneath each experiment (RAW\_DATA) directory (i.e.,
\code{ROOT/STUDY/group/exp_YYYY_MM_DD_HH}), we have the following structures:
\begin{verbatim}
redcap_data/
    team_data.csv 

baseline_tasks/
    affective/
        individual_<cat>_<participantID>_<timestamp>.csv
        individual_<cat>_<participantID>_<timestamp>_metadata.json
        team_<timestamp>.csv
        team_<timestamp>_metadata.json
    finger_tapping/
        <timestamp>.csv
        <timestamp>_metadata.json
    ping_pong/
         competitive_0_<timestamp>.csv
         competitive_0_<timestamp>_metadata.json
         competitive_1_<timestamp>.csv
         competitive_1_<timestamp>_metadata.json
         cooperative_0_<timestamp>.csv
         cooperative_0_<timestamp>_metadata.json
    rest_state/
        <timestamp>.csv

lsl/
    block_1.xdf
    block_2.xdf
       
minecraft/
    MinecraftData_Trial-T00073_ID-<fancy_string>.metadata
    MinecraftData_Trial-Training_ID-<fancy_string>.metadata

<cat>/
    audio/
    face_images/
    pupil_recorder/
    redcap_data/
    screen_shots/

testbed_logs/
    asist_logs_<timestamp>/

tmp/
    
data_inventory.log
data_inventory.run
time_difference.txt

trial_info.json
\end{verbatim}

\subsubsection{Description of the files.}

\noindent 
Excluding log files, debugging, and other infrastructure files, the format and
the data for each file listed
above is detailed as follows: \\


\begin{description}
\item{redcap\_data/team\_data.csv}\\
describe the format and the data

\item{baseline\_tasks/affective/individual\_<cat>\_<participantID>\_<timestamp>.csv}\\
describe the format and the data

\item{baseline\_tasks/affective/individual\_<cat>\_<participantID>\_<timestamp>\_metadata.json}\\
describe the format and the data

\item{baseline\_tasks/affective/team\_<timestamp>.csv}\\
describe the format and the data

\item{baseline\_tasks/affective/team\_<timestamp>\_metadata.json}\\
describe the format and the data

\kobus{Many more files}
\end{description}







