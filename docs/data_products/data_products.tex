

\documentclass[12pt,letterpaper,oneside,openany]{article}

\usepackage{graphicx}
\usepackage{authblk}
\usepackage{hyperref}
\usepackage{upquote}
\usepackage{verbatim}
\usepackage{url}
\usepackage{subcaption}
\usepackage{amsmath}

\usepackage{listings}
\lstset{language=bash,
 % How/what to match
    sensitive=true,
    % Border 
	frame=single,
    % Put extra space under caption
    belowcaptionskip=1\baselineskip,
    % Colors
    backgroundcolor=\color{sbase3},
    basicstyle=\small\color{sbase00}\ttfamily,
    keywordstyle=\color{scyan},
    commentstyle=\itshape\color{sbase1},
    stringstyle=\color{sblue},
    numberstyle=\scriptsize\color{sviolet},
    identifierstyle=\color{sbase00},
    % Break long lines into multiple lines?
    breaklines=true,
    tabsize=2
} 

\usepackage[normalem]{ulem}  % provides strikeout with \sout{...}

\usepackage[T1]{fontenc}
\usepackage[scaled=0.85]{beramono}

\newcommand{\code}[1]{\lstinline{#1}}

\usepackage{xcolor}

\def\marrow{{\marginpar[\hfill$\longrightarrow$]{$\longleftarrow$}}}

\def\kobus #1 {{\textcolor{red}{ \sc \newline\newline Kobus says: }{\marrow\sf #1 \newline\newline}}}
\def\jinyan #1 {{\textcolor{green}{ \sc \newline\newline Jinyan says: }{\marrow\sf #1 \newline\newline}}}
\def\clay #1 {{\textcolor{blue}{ \sc \newline\newline Clay says: }{\marrow\sf #1 \newline\newline}}}
\newcommand{\Adarsh}[1]{{\color{Maroon} \sc Adarsh:} {\sf #1}}


% I am not sure what these do for us, and when. Maroon and RoyalBlue should be
% defined by package xcolor.
%
\definecolor{webgreen}{rgb}{0,.5,0}
\definecolor{webbrown}{rgb}{.6,0,0}
\definecolor{Maroon}{cmyk}{0, 0.87, 0.68, 0.32}
\definecolor{RoyalBlue}{cmyk}{1, 0.50, 0, 0}
\definecolor{Black}{cmyk}{0, 0, 0, 0}
\definecolor{shadecolor}{gray}{0.9}
\definecolor{sbase03}{HTML}{002B36}
\definecolor{sbase02}{HTML}{073642}
\definecolor{sbase01}{HTML}{586E75}
\definecolor{sbase00}{HTML}{657B83}
\definecolor{sbase0}{HTML}{839496}
\definecolor{sbase1}{HTML}{93A1A1}
\definecolor{sbase2}{HTML}{EEE8D5}
\definecolor{sbase3}{HTML}{FDF6E3}
\definecolor{syellow}{HTML}{B58900}
\definecolor{sorange}{HTML}{CB4B16}
\definecolor{sred}{HTML}{DC322F}
\definecolor{smagenta}{HTML}{D33682}
\definecolor{sviolet}{HTML}{6C71C4}
\definecolor{sblue}{HTML}{268BD2}
\definecolor{scyan}{HTML}{2AA198}
\definecolor{sgreen}{HTML}{859900}


\PassOptionsToPackage{dvipsnames}{xcolor}
    \RequirePackage{xcolor} % [dvipsnames] 

\PassOptionsToPackage{pdftex,hyperfootnotes=false,pdfpagelabels=true}{hyperref}
    \usepackage{hyperref}  % backref linktocpage pagebackref

\hypersetup{%
    colorlinks=true, linktocpage=true, pdfstartpage=3, pdfstartview=FitV,%
    breaklinks=true, pdfpagemode=UseNone, pageanchor=true, pdfpagemode=UseOutlines,%
    plainpages=false, bookmarksnumbered, bookmarksopen=true, bookmarksopenlevel=1,%
    hypertexnames=true, pdfhighlight=/O,%nesting=true,%frenchlinks,%
    urlcolor=webbrown, linkcolor=RoyalBlue, citecolor=webgreen, %pagecolor=RoyalBlue,%
    pdftitle={IVILAB Manifesto},%
    pdfcreator={pdfLaTeX},%
    pdfproducer={LaTeX}%
}   

\def\sectionautorefname{\S}
\def\subsectionautorefname{\S}


%%% % Most of this preamble stuff is copied from the IVILAB preambles
%%% 
%%% \documentclass[12pt,letterpaper,oneside,openany]{book}
%%% 
%%% \usepackage{graphicx}
%%% \usepackage{authblk}
%%% \usepackage{hyperref}
%%% \usepackage{upquote}
%%% \usepackage{verbatim}
%%% \usepackage{url}
%%% \usepackage{subcaption}
%%% \usepackage{amsmath}
%%% 
%%% \newcommand{\code}[1]{\lstinline{#1}}
%%% 
%%% \usepackage{xcolor}
%%% 
%%% \def\marrow{{\marginpar[\hfill$\longrightarrow$]{$\longleftarrow$}}}
%%% 
%%% \def\kobus #1 {{\textcolor{red}{ \sc \newline\newline Kobus says: }{\marrow\sf #1 \newline\newline}}}
%%% \newcommand{\Adarsh}[1]{{\color{Maroon} \sc Adarsh:} {\sf #1}}
%%% 
%%% \def\sectionautorefname{\S}
%%% \def\subsectionautorefname{\S}

\title{ToMCAT data products}

\author{Adarsh Pyarelal, Rick Champlin, Paulo Soares, Eric Duong, Caleb
Jones, Chinmai Basavaraj, Kobus Barnard}

\date{\today}

\begin{document}
\maketitle
\setcounter{page}{2}

\section{Overview}

The main experiment involves a three person team participating in one or two
search and rescue (SAR) missions simulated in the Minecraft environment. 
Before participants execute this experiment, they individually attend a
presession where 1) we measure their head circumference to enable setting up brain data caps
before they arrive at the main session; 2) we ask them to do XXX to calibrate
their speech; 3) XXX. The main session typically occurs a few days to few weeks
later. 

The main experiment collects data from multiple phases, consisting of a training
session lasting for XXX minutes, three baseline activities, and one or two SAR
missions, depending on whether the experiment has gone well enough that there is
time for a second SAR mission.  The first baseline activities is XXX. The second
baseline is XXX.  The third baseline is XXX.  During all these phases, for each
participant, we collect a variety of physiological measures including EKG, EEG,
and fNIRS, Minecraft game data, audio recordings, eye tracking data, face data,
and what is on their screen. 

\section{Raw data}

In what follows, upper case indicates
a placeholder for a more specific string (i.e., variables), whereas lower case or mixed case
indicates actual strings (i.e., verbatim).
We break the directory structure for raw ToMCAT data into three parts
\code{ROOT\/EXPERIMENT\_INFO\/RAW\_DATA}. 
As the experiment is run, data is written to the LangLab linux computer called
``cat''. The root directory (ROOT) on cat is
\code{/data/cat}. The data gets mirrored onto the LangLab linux computer called
``tom'', by the script \code{sync\_tom\_and\_cat}, which is called by the script 
\code{pull\_tomcat\_data}. Ideally, \code{sync\_tom\_and\_cat} should also be
called from the main driver script as soon as the experiment is over, but
currently we do not do this. 

The
script \code{pull\_tomcat\_data} transfers the data to the IVILAB machine
\textit{i03.cs.arizona.edu}, and makes two backups of it. 
Ideally, we will create one or two off-site backups, but we do not do this yet.
The data is is written to
\code{/tomcat\_raw\_NNN} where NNN is 1, 2, 3, or 4, and backed up to 
\code{/tomcat\_raw\_NNN\_B1} and \code{/tomcat\_raw\_NNN\_B2}.  
The script \code{pull\_tomcat\_data} then makes links to those multiple data
locations from \code{/tomcat/data/raw} and provides access via NFS to the compute
servers \textit{laplace.cs.arizona.edu} and \textit{gauss.cs.arizona.edu}.
Thus, on those IVILAB machines, ROOT is \code{/tomcat/data/raw}.

The directory structure pattern for EXPERIMENT\_INFO under the root directory is
\code{FACILTY/experiments/EXPERIMENT\_NAME}. For this experiment, FACILITY is
\textit{LangLab}, and EXPERIMENT\_NAME is \textit{study\_3\_pilot}.
This experiment name is a bit misleading, but makes senses as this experiment
gradually morphed from an initial pilot study to a real one as we developed the
system, but all data is informative. 

RAW\_DATA has two subdirectories, ``presession'' and ``group'', containing data
from the presessions and main experiments separately. In both cases, we put the
data from one experimental instance into a directory named
\code{exp\_YEAR\_MM\_DD\_HOUR}. Since we only run one session at a time, and
they last from most of an hour to over three hours, hourly time resolution
suffices to disambiguate them. To further clarify, on the IVILAB cpu servers, data for the first precession
experimental session is in:\\
\code{/tomcat/data/raw/data/LangLab/experiments/study\_3\_pilot/presession/exp\_2022\_09\_14\_07}\\
However, this might be reported differently because of the linking described
above as:\\
\code{/tomcat_raw_1/data/LangLab/experiments/study\_3\_pilot/presession/exp\_2022\_09\_14\_07}\\
Similarly, the data for the first group session, which would occur after all
three participants had done a presession \kobus{But this is not the case!!} is in:\\
\code{/tomcat/data/raw/LangLab/experiments/study\_3\_pilot/group/exp\_2022\_04\_01\_13}\\

\subsection{Raw data structure for presessions.}

\subsection{Raw data structure for group sessions.}


\section{Exported data formats}







\end{document}





