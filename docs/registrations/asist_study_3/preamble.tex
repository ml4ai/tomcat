\documentclass[oneside,9pt]{memoir}

% Colors
\usepackage[dvipsnames]{xcolor}
\usepackage[scaled=0.8]{beramono}

% hyperref
\usepackage[%
    colorlinks=true,
    linktocpage=true,
    breaklinks=true,
    linkcolor=NavyBlue,
    urlcolor=NavyBlue,
    citecolor=NavyBlue
]{hyperref}
\def\chapterautorefname{§}
\def\sectionautorefname{§}
\def\subsectionautorefname{§}
\def\subsubsectionautorefname{§}

\setcounter{secnumdepth}{3}


% Fonts
\usepackage{fontspec}
\setmainfont[Numbers = {OldStyle, Proportional}]{TeX Gyre Pagella}

\usepackage{amsmath}
\DeclareMathOperator*{\argmax}{arg\,max}
%\usepackage{amssymb}
\usepackage{unicode-math}
\setmathfont{TeX Gyre Pagella Math}

\usepackage{algorithm}
\usepackage[italicComments=false]{algpseudocodex}

\newcommand{\algorithmautorefname}{Algorithm}

% Typography
\usepackage[tracking]{microtype}

% Images
\usepackage{graphicx}

% ======================================================================
% Memoir package - layout and styling
% ======================================================================

% The calc package is required for calculating readable text widths
\usepackage{calc}

% Set outer and spine margins A wide right margin is chosen both for legibility
% of the typeblock and for tight integration of marginfigures and margin
% footnotes.

% Calculate widths in pts
\setlxvchars[\normalfont\normalsize] % about 66 characters per column
\setxlvchars[\normalfont\footnotesize] % about 45 characters per column

% Set left and right margins
\setlrmarginsandblock{1.15in}{3.5in}{*}
% Set upper and lower margins
\setulmarginsandblock{1.1in}{1.1in}{*}

% Set properties of margin notes, sidecaptioned floats, and footnotes in the
% margin.
\setmarginnotes{0.2in}{2.15in}{2\onelineskip}
\setsidecaps{0.2in}{2.15in}
\sidecapmargin{outer}
\renewcommand*{\sidecapstyle}{\normalfont\footnotesize}
\setsidecappos{c}


% Set footnotes in the margin rather than at the foot of the page
\footnotesinmargin
\setsidefeet{\marginparsep}{1.9in}{0.2in}{0pt}{\flushleftright\footnotesize}{*}

% Integrate the counters of the sidefootnotes and footnotes in margin.
\letcountercounter{sidefootnote}{footnote}
\setlength{\footmarkwidth}{0em}
\setlength{\footmarksep}{-\footmarkwidth}
\setlength{\footparindent}{1em}
\sideparmargin{outer}

\renewcommand*{\sideparfont}{\color{Maroon}\emphshape}
\renewcommand*{\sideparvshift}{2\baselineskip}
\marginparmargin{outer}

% Style the entries in the Table of Contents
\renewcommand*{\cftchapterfont}{\scshape\MakeTextLowercase}
\renewcommand*{\cftpartfont}{\color{Maroon}\scshape\MakeTextLowercase}
\captionstyle[\centering]{\footnotesize}
\captionnamefont{\footnotesize\color{Maroon}}

%% Bringhurst chapter and head styles with a Pedersen-type chapter number
\makechapterstyle{bringhurst}{%
	\renewcommand{\chapterheadstart}{} 
	\renewcommand{\printchaptername}{} 
	\renewcommand{\chapternamenum}{} 
	\setlength{\midchapskip}{15mm}
	\renewcommand*{\printchapternum}{%
        \begin{marginfigure}[0pt]
          \resizebox{!}{\midchapskip}{\color{Maroon}\emph{\thechapter}}
        \end{marginfigure}
      }
	\renewcommand{\afterchapternum}{} 
	\renewcommand{\printchaptertitle}[1]{%
	  \raggedright\Large\scshape\MakeLowercase{##1}}
	\renewcommand{\afterchaptertitle}{%
	  \vskip\onelineskip \hrule\vskip\onelineskip}
}
\setlength{\cftsubsectionindent}{0.6in}
\chapterstyle{bringhurst}
\headstyles{bringhurst}

\tightlists

% Headers and footers - page numbers, section headings, etc.
\makepagestyle{tufte}
\createmark{chapter}{left}{nonumber}{}{}
\makeoddhead{tufte}{}{}{\scshape\MakeTextLowercase{\leftmark}~~|~~\thepage}
\makeevenhead{tufte}{\thepage~~|~~\scshape\MakeTextLowercase{\rightmark}}{}{}
 \makerunningwidth{tufte}[8in]{8in}
\aliaspagestyle{chapter}{empty}
\nouppercaseheads
\pagestyle{tufte}

\linespread{1.1}
\checkandfixthelayout

% Bibliography management
\usepackage[%
    backend=biber,
    style=numeric-comp,
    sorting=none,
    natbib
]{biblatex}
\addbibresource{../bibliography.bib}

\usepackage[backgroundcolor=white, bordercolor=white, textsize=small]{todonotes}

% Added by Paulo Soares
\newcommand{\ps}[1]{\todo{\textcolor{Cyan}{PS: #1}}}
\newcommand{\ap}[1]{\todo{\textcolor{Maroon}{\small AP: \emph{#1}}}}
\newcommand{\kobus}[1]{\todo{\textcolor{Red}{Kobus: #1}}}
