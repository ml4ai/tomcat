\chapter{Entrainment detection}
\label{ch:entrainment}

\textbf{Meghavarshini Krishnaswamy, Andrew Wedel, Adam Ussishkin, Cheonkam
Jeong, Adarsh Pyarelal} 

\section{Introduction}

Entrainment (also known in the literature as `synchronization',
`coordination'\footnote{Note that this is distinct from the mathematical
definition of coordination proposed in \autoref{ch:pgm}.}, or `alignment') is
the adaption of verbal and non-verbal actions by conversation partners to more
closely resemble one another \citep{borrie2014}. Its role in communication has
been described as ``key for supporting important pragmatic aspects of
conversation, including taking turns, interaction smoothness, building rapport,
fostering social bonds, and maintaining interpersonal relationships"
\citep{borrie2019}.  A time-sensitive cooperative task utilizing verbal
communication would require participants to communicate efficiently -
entrainment is one way in which humans do this. This makes entrainment a useful
metric for assessing the degree of cooperation among teammates, as well as a
useful means to assess members' sentiments towards each other, tracking their
confidence in the task goals and plans, and identifying team cohesion and
bonding over time.

In speech, entrainment has been observed and analysed using speech rhythm and
timing, pitch, MFCCs and formant analysis
\citep{reichel2018prosodic,borrie2019syncing}. Speech entrainment occurs in
correlation with entrainment at other linguistic levels such as an increase in
shared vocabulary and sentence structures \citep{rahimi2017entrainment}.  While
entrainment in multi-party conversations\footnote{That is, conversations with
more than two participants.} has been researched at both the lexical and syntax
levels, work on vocal entrainment has mostly focused on two-party
conversations. Work on vocal entrainment is further limited by the practical
difficulties in identifying speech similarity due to the process of entrainment
from similarities arising from physical factors such as people having similar
vocal characteristics, or sharing the same recording channel \citep{nasir2020}.

Entrainment is assessed at two scales by analysing changes in pairs of adjacent
utterances spoken by different conversation partners (local entrainment) and by
analysing changes in utterances by same speaker at the beginning and end of the
discourse (global entrainment). Local entrainment checks for inter-speaker
convergence, while global entrainment assesses if speakers have shifted from
their own baseline vocalic characteristics after a period of interaction. 

Recent research on vocal entrainment has shifted from regression-based
analysis to encoding-based neural networks in order to better model the
complex and non-linear relationships between vocalic features in both
entrainment and disentrainment contexts. In addition to improved feature
modelling, neural networks show better performance in distinguishing
information relevant to entrainment, while ignoring other factors that
cause unrelated feature similarity \citep{nasir2020}.

We propose to design and implement an automated multi-party vocal entrainment
detection system. We will begin by replicating the work of \citet{nasir2020},
followed by extending the system to handle vocal entrainment between more than
two participants. This component will be tested offline for Study-3 and
deployed live in Study-4.

\section{Approach}

Automated entrainment detection typically compares featural similarities
between adjacent inter-pausal units or IPUs (in this case, speech units
separated by a 50ms pause) of two participants during one part of a discourse
to those from different parts of the discourse. If entrainment is taking place,
an IPU should show greater featural similarity to the immediately preceding IPU
from the conversation partner than to one in a different part of the discourse.
The automated system therefore needs to successfully identify adjacent IPUs
(the \emph{entrainment context}) across paired participants and compare them
to two distant IPUs. For a multi-party conversation, the model needs to
differentiate between IPUs in an entrainment context and IPUs that are
between the same participants who are not addressing each other.

For this study, we replicate the feed-forward encoder, i-vector modeling and
`triplet network-based entrainment distance measures' for entrainment detection
proposed and designed by \citet{nasir2020}. First, i-vectors are
extracted for each IPU spoken by each speaker. Next, triplets are created by
taking two adjacent IPUs, by different
speakers (positive match for entrainment), and combining them with a randomly
chosen IPU by the second speaker from a different location in the discourse
(negative match for entrainment, but one that shares acoustic similarities with
the given IPU). These are then used to train the triplet network
model, such that in the embedding space, the positive match for entrainment is
closer to the IPU being assessed than the negative match \cite{hoffer2015deep}.

    For the entrainment detection and other qualitative assessments, we will
    utilize the following data and labels from Study-3:

            \begin{itemize}               
                   \item Audio recording
                    Transcript
                   \item Participant demographic information
                   \item Self-evaluation
                   \item MinecraftEntity\_Observation\_Asr\_Speechanalyzer
                   \item MinecraftEntity\_Observation\_Audio\_Speechanalyzer
                   \item MinecraftEntity\_Event\_Dialogue\_Event\_Dialogagent
            \end{itemize}

\section{Evaluation}

We will evaluate the performance of the model resulting from replicating
\citet{nasir2020} (i.e., trained on the Fisher corpus\cite{cieri2004fisher}) on
the Study-3 data. The evaluation consists of a binary classification task to
distinguish between the entrainment context and two randomly chosen utterances
\footnote{These utterances need not belong to the same speaker.}. Accuracy
scores from the classification task will be utilised as a metric of
performance. Since the teams in Study-3 engage in multi-party conversations, we
will set up new baselines for entrainment detection that can account for
differences in team dynamics and demographics (the training data ), in the
volume of utterances from each speaker, and volume of utterances directed at
different speakers.

We will also explore methods to evaluate the following two types of
entrainment:

\begin{itemize}

    \item \emph{Local entrainment}: Are there observable similarities between
        adjacent utterances by two speakers than utterances that are not
        adjacent?

    \item \emph{Global entrainment}: Do speaker characteristics of conversation
        partners change after a given period of communication?

\end{itemize}

In order to account for ASR errors, we will create a corpus of human-generated
`gold' transcriptions. A subset of this corpus will be used to
identify IPUs, in order to assess any differences in performances due to ASR
errors. The speech data will also be annotated to identify
environmental noise or other distortions, so as to account for qualitative differences
between the recording conditions of the pristine training data from the Fisher
corpus and the potentially noisy language data from Study-3, where we have
little control over the recording setups due to the remote experimentation
protocol.

\section{Other Qualitative Assessments}

Since speech entrainment in conversational setups is a feature of speaker
bonding and closeness, it is useful to assess if it co-occurs with positive
emotions in spoken statements, lower stress, and higher confidence in the plan
and team actions in a co-operative goal-oriented setting. We will utilise the
human-generated gold transcriptions as well as manual annotations for emotion
(see \autoref{ch:sentiment_analysis}) and dialogue acts (see
\autoref{ch:da_classification}) to qualitatively assess if entrainment
co-occurs with positive emotions and the acts of sharing information, as well
as the relationship between global entrainment and levels of stress and
uncertainty.  We will also examine if the directionality of entrainment
(towards or away from any of the participants) is affected by members'
personality or socio-cultural differences, in order to account for natural
differences known to impact conversation styles.

