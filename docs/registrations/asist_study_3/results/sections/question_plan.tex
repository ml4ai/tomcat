\chapter{Question-Asking and Plan Inference}
\label{ch:question_plan}
\textbf{Salena Torres Ashton, Stephen Kim, Loren Rieffer-Champlin, Liang Zhang,
Adarsh Pyarelal, Clayton Morrison}
%% Start on line 115

\section{Introduction}
\label{sec:question_plan_intro}

\subsection{Significance of Question-Asking in SAR Scenarios}

When we gather information or clarify ambiguity, we often do so through
question-asking and question-answering. This mode of communication reflects our
intention, strengthens relationships, increases productivity, reduces risk, and
creates an equitable environment \citep{rothe_lake_gureckis_2018, alaimi_2020}
and between dialog partners. In the SAR scenario implemented for ASIST Study 3,
teams of human players engage in cooperative behavior to search for, stabilize,
and rescue victims within a collapsed building. As human players verbalize
plans, make suggestions, or tell each other what to do, they also ask questions
to infer hidden goals or intentions. Question-asking enables teammates to
reduce their individual knowledge asymmetry.

%Not understanding a goal or intention can be seen in the designs of AI agents
%that may have been programmed with researcher bias. If an AI agent's purpose is
%to assist humans in a search and rescue (SAR) scenario, the programmer and
%researcher must justify their own domain knowledge of SAR communications to
%curb potential bias toward academic interests when programming uncertainty,
%actions, or intent of players. Understanding SAR communications enables
%programmers to design a realistic and scalable domain definition. Not doing so
%can result in flawed domains that predict human player and AI agent action no
%better than chance.

Question-asking and information-sharing is vital to SAR missions.  The INSARAG
marking system\footnote{International Search and Rescue Advisory Group. FEMA
    has a similar marking but is only used in the United States, whereas
    INSARAG is used internationally.} is designed to immediately answer the
    most frequently-asked questions a rescuer will have in an urban SAR
    disaster: Go or No-Go, number of live victims, number of dead victims,
    number of unaccounted-for victims, hazards, building floor references,
    cleared areas of building, and cleared building \citep{insarag_2022}. SAR
    rescuers use radio, sound horns, Morse code and direct communication with
    victims and each other.

This registration investigates the relation between question-asking in natural
language discourse and player goals through goal inference. Using procedural
Grounded Theory to annotate video observations of teams who play Minecraft in
an urban search and rescue scenario, we map players' utterances to their actions and
intentions. The actions and intentions are then mapped into action and task
definitions in the HDDL declarative language\footnote{See
    \autoref{ch:plan_recognition}
    for a description of HDDL and our plan
    recognition approach. While simple human goals can be represented with a
    classical planning approach, we believe that complex goals with constraints or
    multiple levels of abstraction are be best represented by a 'hierarchical
    task network', (HTN). Technically, the
    plans produced by HTN planners can also be represented with flat lists.
    However, in this section, we use the term `plan' to refer to the actual
`plan tree' that contains the task decompositions as well, rather than just the
plan alone.}

Our analysis shows that teams that ask more questions at a more consistent rate
perform better than teams that ask fewer questions, ask more `how-to'
questions, or use questions simply as a polite way to give orders. We conclude
with suggestions for further research on knowledge engineering and
representation of tasks, goals, and hidden intentions.

\subsection{Literature}

\begin{enumerate}
    \item How do spoken questions reveal another person's plan or intent?
    \item Can people listen to spoken questions and accurately decide if the other person's plan is sequential or hierarchical? Can the distinction between such structures improve predictive performance for Artificial Social Intelligence (ASI) agent intervention?
\end{enumerate}

Artificial intelligence (AI), information retrieval, and natural language
processing yield exciting research into question-\emph{answering}, relevant
document retrieval, theory of mind (ToM), and human-computer interaction.
However, the literature is sparse within the  \emph{intersection} of artificial
intelligence, psycholinguistics, indirect speech acts, negative knowledge
modeling, ToM, and question-\emph{asking}. For example,
\citet{hawkins_goodman_2017} address this sparsity and the trend of past
research to focus on the optimization of question-answering, not
question-asking.


When we ask questions, we are not seeded with a set of questions or handed a
set of `correct and incorrect' questions. We do not always prefer questions
with literal, logical answers \citep{rothe_lake_gureckis_2017, clark_1979}.
However, most research so far has focused on answer sets or query matching,
preference for concise questions and answers. In reality, people ask open-ended
questions, sloppy half questions, use questions to politely tell others
what to do, and meander in speech without purpose \citep{brown_1980}. We ask
questions to express goals.

Goals reflect a desired endpoint or world state. A goal can be motivated by
belief, desire or intent. People ask questions that either express a goal or
infer a hidden goal. A question is the interpretation and response of a hidden
(or uttered) goal to be discerned by another person, typically a dialog partner
\citep{hawkins_goodman_2017}. Indirect speech acts and question-answer dialog
reflect \emph{information asymmetry}: a questioner has a goal but needs
information, while an answerer has information but does not know the
questioner’s goal. Missing or contradictory information naturally leads to a
knowledge goal and the generation of questions. The person is then motivated to
ask a question \citep{alaimi_2020}.

Intentions demonstrate action choice to reach that goal-- it is reflected in
the way a question is to be asked. A questioner believes that the listener's
knowledge base has information that can help the questioner achieve some goal.
The type of questions asked will depend on context, social inference, and other
signals. The key is to decouple\footnote{\citet{hawkins_goodman_2017} was
limited to epistemological questions and cooperative behavior.} the inferred
goal from the explicit meaning of the question \citep{hawkins_goodman_2017}.
This redefinition of a \emph{question} enables models to account for context
and avoid assumptions.

Questions can be viewed as reflective of hidden goals or intents, but they can
also be viewed as a way to provide or reinforce knowledge \citep{alaimi_2020,
ray_2001}. Question-askers want the listener to directly understand
their goal, or to discern a hidden goal, or both --- then to give appropriate
responses (however that appropriate response may be for the context). When
question-askers do not receive appropriate answers from listeners, both dialog
partners use conversational implicature and politeness to reach a mutual
understanding. They will minimize potential transaction and opportunity costs
in conversation to reach mutual understanding.

\section{Approach}
\label{sec:question_plan_approach}

We assume that questions have hidden goals and infer
plans. As teams ask more questions of each other, human team ToM converges
toward cooperative behavior.  We will investigate whether question-asking is
associated with \emph{team planning}, defined as a set of goals, strategies, or
tasks that are executed.  We define \emph{coordination} as behaviors and
utterances to create a common plan or strategy\footnote{Note that this
definition is distinct from the mathematical definition of coordination
proposed in \autoref{ch:pgm}}. We define \emph{cooperation} as team behaviors
that implement an already-agreed up on plan.

To capture hidden goals, inferred plans, and patterns that may represent human
ToM, we will annotate six ASIST Study 3 Spiral 2 pilot video observations and
six HSR ASIST Study 3 videos released between March 29 and May 5, for a total
of at least twelve videos. These videos are of three distinct missions for each
team.

Two human annotators will code all uttered questions between teammates within
the Minecraft SAR scenario.  We use the qualitative coding procedure known as
Grounded Theory \citep{corbin_strauss_2015}.

More specifically, and as defined by \citet{saldana_2021}, we will use a
Grounded Theory Process Coding for a state or action across some interval of
time. These \emph{grounded-in-data} labels are known as \emph{concept-level}
labels, which are the smallest pieces of data that encode a question-asking
phenomena of interest. We also use this methodology to investigate the
connectivity and causality of each concept label to discover possible
relationships between presence or absence of team actions, interactions,
conditions, and consequences of question-asking. Densely-connected concept
labels suggest subcategories and categories. Sparsely-connected labels will not
be discarded --- rather, they will be used to consider variability within
patterns and categories that emerge. In cases where questions have co-reference
or other contextual dependencies, only that direct dependency will be coded for
local semantic meaning.

We make the following considerations when creating codes:

\begin{itemize}
    \item Frequency will not dictate importance, causality, or connectivity of
        a concept.
    \item Each question will have at least one annotation and up to four
      annotations:
    \begin{itemize}
        \item Primitive actions (ground truth). Ex: breakRubble,
          requestStabilizedVictimCarry.
        \item Abstraction Levels of actions (of primitive actions) Respective
          examples: respondRubbleRequest or createVictimAccess, collaborateStabilizedVictim
    \end{itemize}
    \item Labels will be stemmed and minimally normalized.
    \item Capturing the phenomena of question-asking across time, between any subset of a team, between the same team across the two different missions.
\end{itemize}


To avoid annotator and researcher biases and any \emph{a priori} belief on
which team strategies may be used, concept and category labels are not
pre-determined. Inter-annotator agreement must reach a Kappa Score of 80\% or
higher. This also gives a more solid, grounded analytical meaning to any
emergent categories.

After the development of the labels and taxonomy, the investigation of team ToM
and question-asking will be extended to additional videos. When all concepts,
subcategories, categories can reasonably explain the phenomena of the video
observations, one or two super-categories, \emph{theories of team plan}, will
emerge. We currently assume that a theory of team plan would have greater
predictive power and ToM inference potential.


\section{Evaluation}
\label{sec:question_plan_eval}

\subsection{Label Development and Kappa Score}

The results of this study show that teams who ask more questions at a more
consistent rate lead to better team performance than teams who ask fewer
questions. When teammates ask each other questions, they engage with each other
more often, evidenced by the rescue of critical victims (which require
teamwork) as opposed to regular victims (which can be done individually).

To capture hidden goals, inferred plans, and patterns, twelve ASIST Study 3
video observations were annotated using a
bottom-up Grounded Theory qualitative coding method to investigate connections
between a question-asker's intention and words spoken. This rigorous theory
ensures that the labeling is not biased, as it uses unsupervised labels and
does not lend itself to the proof or disproof of prior researcher belief in
Minecraft SAR player strategy.  We were able to capture player intention, team
strategy, and begin preliminary quantitative analysis. Videos were selected
from no-human intervention videos and two independent annotators\footnote{Pilot
Data: Salena Ashton and Stephen Kim, MS Information Science; Real Data: Salena
T. Ashton and Loren Rieffer-Champlin, both PhD Students, School of Information,
University of Arizona; } annotated the videos with unconstrained, unsupervised
labels and were free to create any label desired to code a player's question.

We disambiguated these unsupervised labels using lexical definitions, which led
to extensive documentation of words used to capture specific phenomena in
observations. This led to Theoretical Saturation\footnote{Theoretical
saturation is the point when few additional labels are created, regardless of
fully-autonomous annotators' ability to create new ones. Saturation
demonstrates that the existing label schema fully capture the events and
phenomena observed.} and the labeling schema converged to 21 verb labels, 24
noun labels, 25 modifier labels. Without replacement, these components can
yield 12,600 possible labels, yet we demonstrated theoretical saturation
through the independent construction of less than 100 labels.

We achieved an unweighted Cohen's kappa coefficient\footnote{Grounded Theory is robust in its
unsupervised labeling approach is because annotators have complete autonomy.
Because the sample size is small, the theoretical probability of two matching
labels is even smaller, and only two annotators worked on the real data, there
is no current justification for using Scott's pi or a weighted Kappa. This will
change as further research scales.} agreement of 0.892. These 100 resulting labels were analyzed for
emergent categories and preliminary patterns. Label frequency \emph{did not}
dictate importance.

\section{Results}

Data visualization and preliminary analysis focus on three
key features: \textit{questionLabel}\footnote{Qualitative code representation of the
question utterance and no further context except for co-reference resolution.
Some questions were spoken as statements with inflection. Other questions were
actually demands shrouded with politeness. All three types of questions were
coded for this investigation. All questions asked are assumed to be uttered
with cooperative intent.}, \textit{abstractLabel}\footnote{Code representation of
player intention. This was captured by taking the question into context. For
example, if the question were represented by 'requestBreakRubble', the intent
behind the question could have been 'collaborateCriticalWake' or
'navigateLocation'. AbstractLabel is not the representation of 'why' a
question had been asked; it is the 'intention' of the player who asked the
question.}, and \textit{utterance}, the word of interest uttered in a question. From
the emergent categories, specific words were chosen that did or 'did not'
represent player interaction\footnote{For example: 'ask', 'request',
'suggest', and 'clarify' are all actions that show the goal of
obtaining information, but 'clarify' asks for additional information after the
initial question, showing two-way dialog, 'request' is an initiation of
commitment to another player with the optional accept/ reject response, showing
two-way dialog. 'Suggest' also initiates like 'request' but does not give the
option to accept or reject, making the question and dialog a one-way
discussion. 'Ask' is a one-way dialog utterance to obtain information.}.


%%%%%%%%%%%%%%%%%%%%%%%%%%%%%%%%%%%%%%%%%%%%%%%%%%%%%%%%%%%%

Teams that asked more questions and worked together generally saved more
critical victims and scored higher points. While it is possible for teammates
to rescue critical victims without verbalizing any type of request, it is
inefficient due to the time constraint of the mission (15 minutes). Teams also
scored higher when more of their questions were about victims instead of
location or navigation strategies.

% Question Timing
\begin{figure}
    \centering
    \fbox{\includegraphics[width=0.9\textwidth]{../images/QuestionTiming_STA.pdf}}
    \caption{%
        Time is displayed as color, where dark gray is at the beginning of the
        mission and red is toward the end of the mission. Team 218 (videos 635
        and 636) asked the fewest questions, scored the lowest combined points,
        and asked 66\% of questions toward the end of the mission. Teams who
        score poorly tend to ask fewer questions that focus on collaboration or
        planning-- and more questions that ask specific task-oriented items
        toward the end of each mission. Team 3 (videos 637 and 638) asked the
        most questions and asked questions at a \emph{consistent rate}
        throughout their missions. Teams who score higher tend to ask questions
        at a consistent rate, enabling the exchange of information, building of
        rapport and team trust, and setting a social convention of asking each
        other for collaboration or help.
    }
\end{figure}

% Information from figure I deleted for more space in this paragraph:

When a player asked a question about destination, 'requestDestination', the
player's intent (represented by abstractLabels, also known as categorical
labels within the Grounded Theory framework) is more than 80\% likely to be
waking a critical victim. When a player intended to wake a critical, they were
55\% likely to ask another player to join them ('requestDestination'). This
observation is trivial because the game rules state that at least two people
are needed to wake a critical. However, when the condition was flipped, a
player's intention is more than 80\% likely to be waking a critical when
'requestDestination' is asked. When teams collaborate to wake a critical
victim, they request or clarify --not suggest or tell-- a teammate to come to a
particular destination. ('Clarify' and 'request' infer two-way dialog of
commitment creation whereas 'direct' or 'suggest' are creation commitments or
demands that do not give the listener a chance to respond.

Teams require more collaboration and effort to save critical victims than when
saving regular victims; those who ask more questions invite team collaboration
and tend to score higher than teams who do not work together. The most common
types of questions that teammates asked, given that they scored better than
other teams, were question that invited a response, often starting with: "do",
"can", "what", and "which." Common questions which did \textit{not} result in
high scores tended to start with: "have", "is", and "were." These questions
were more about information-seeking than team collaboration.

Players used a non-trivial amount of imperative statements with added
inflections as if they were requests of collaboration. For example, "Take this
victim, will you?" does not invite the listener to accept or deny this request.
Likewise, "You'll take this victim," with inflection at the end of the sentence
\textit{sounds} like a question, but it is also an imperative.

Previous top-down domain definitions constructed by us for plan recognition
(\autoref{ch:plan_recognition} for this domain yielded results
that were no better than chance when predicting a player's intention or plan.
We believe that the use of non-domain specific word utterances from players
when designing a domain can enhance prediction capabilities. Players who work
together tend to say 'come', 'say' (stemmed), 'type' or 'want.' Less
collaboration included word utterances like: 'go', 'I see', 'there are', and
other phrases that inform but do not invite collaboration. More team
collaboration included word utterances like: 'can you', 'should we', 'you
guys', 'what do you think', 'let's, wake', 'this time'.

Additional results are shown in \autoref{fig:non-domain-specific}
and \autoref{fig:prob-player-intention}.


\begin{figure}
    \centering
    \begin{sidecaption}{%
        Non-domain-specific words that show collaboration include: say, type, or
        come. The \emph{significance} of this figure shows that a priori researcher
        bias can be harmful when designing a domain. For example, our team
        previously believed that words like: navigate, location, victim, room,
        critical and marker were the most important words to look for when
        inferring player intention. This figures shows that is not the case.
    }[fig:non-domain-specific]
    \includegraphics[width=0.9\textwidth]{../images/QuestionUtterances_Link_STA.pdf}
    \end{sidecaption}
\end{figure}


\begin{figure}
    \begin{sidecaption}{%
        Probability of Player Intention, given Utterance Trigram. Stops words
        that are traditionally removed for natural language processing ought to
        be \emph{included} when inferring goals or intentions because they show
        the logical inference of such plans. For example, the word "if" shows a
        conditional statement from one player to another. "If you can..." was a
        common utterance but it was also a polite imperative, not a request or
        question. The phrase "do you guys" and "do we want" do not carry as
        much information about named entities, but they do show ample
        collaboration of teammates.
    }[fig:prob-player-intention]
    \includegraphics[width=0.9\textwidth]{../images/FINAL_1_trigram_Utterances_STA.pdf}
\end{sidecaption}
\end{figure}

\subsection{Discussion}

The ultimate goals of this registration are to investigate player intent, given
a question being asked, and to develop algorithms that extract this intent from
utterance transcriptions without human annotator dependency.

We have discussed how questions can infer explicitly-stated goals, inferred
goals, and how questions are better understood through indirect speech acts. We
then use a Search and Rescue scenario within ASIST's Minecraft Environment to
investigate how teammates ask each other questions in order to make goals.
These goals and intentions were captured with Grounded Theory, annotated by two
completely-autonomous annotators, and reached an unweighted Cohen's kappa
coefficient of 0.892. We then investigated these data through a series of
visualizations, joint and conditional probabilities, and mapping the data to
natural language utterances. We found that teams are more likely to score
higher when they focus on critical victims and collaborate verbally, score
higher when they ask more questions, and ask more questions that require a
response.

Further research is warranted for the investigation of question-asking,
goal-setting, goal-reaching, and team collaboration. As a bonus, we have
discovered that the stop-words not only hold potential research investigation
for the logical flow of such arguments, but can possibly yield new insight into
causal inference and reasoning. The model we built for annotation and labeling
of video observations will extend to ASIST Study 4 videos, regardless of ASI
interventions, changes in domain events, or player leadership emergence.
